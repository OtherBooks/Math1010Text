{\index{multiplicity ! effect on the graph of a polynomial} There is one subtlety about the role of multiplicity that we need to discuss further; specifically we need to see `how' the graph crosses the $x$-axis at a zero of odd multiplicity.  In the section, we deliberately excluded the function $f(x) = x$ from the discussion of the end behaviour of $f(x) = x^{n}$ for odd numbers $n$ and we said at the time that it was due to the fact that $f(x) = x$ didn't fit the pattern we were trying to establish.  You just showed in the previous exercise that the end behaviour of a linear function behaves like every other polynomial of odd degree, so what doesn't $f(x) = x$ do that $g(x) = x^{3}$ does?  It's the `flattening' for values of $x$ near zero.  It is this local behaviour that will distinguish between a zero of multiplicity 1 and one of higher odd multiplicity.  Look again closely at the graphs of $a(x) = x(x + 2)^{2}$ and $F(x) = x^{3}(x + 2)^{2}$ from Exercise \ref{polygraphexercise}.  Discuss with your classmates how the graphs are fundamentally different at the origin.  It might help to use a graphing calculator to zoom in on the origin to see the different crossing behaviour. Also compare the behaviour of $a(x) = x(x + 2)^{2}$ to that of $g(x) = x(x + 2)^{3}$ near the point $(-2, 0)$.  What do you predict will happen at the zeros of $f(x) = (x - 1)(x - 2)^2(x - 3)^{3}(x - 4)^{4}(x - 5)^{5}$?
}
{}