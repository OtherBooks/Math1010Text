{Let $\phi$ be an angle measured in radians and let $P(a,b)$ be a point on the terminal side of $\phi$ when it is drawn in standard position.  Use Theorem \ref{cosinesinecircle} and the sum identity for sine in Theorem \ref{sinesumdifference} to show that  $f(x) = a \, \sin(\omega x) + b\, \cos(\omega x) + B$ (with  $\omega > 0$) can be rewritten as $f(x) = \sqrt{a^{2} + b^{2}}\sin(\omega x + \phi) + B$.
\label{sinusoidexercise2}}
{}
