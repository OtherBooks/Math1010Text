{The population of Sasquatch in Bigfoot county is modelled by \[P(t) = \dfrac{120}{1 + 3.167e^{-0.05t}}\] where $P(t)$ is the population of Sasquatch $t$ years after $2010$.

\begin{enumerate}

\item  Find and interpret $P(0)$.

\item  Find the population of Sasquatch in Bigfoot county in 2013.  Round your answer to the nearest Sasquatch.

\item  When will the population of Sasquatch in Bigfoot county reach 60?  Round your answer to the nearest year.

\item  Find and interpret the end behaviour of the graph of $y = P(t)$.  Check your answer using a graphing utility. 

\end{enumerate}}
{\begin{enumerate}  \item  $P(0) = \frac{120}{4.167} \approx 29$.  There are 29 Sasquatch in Bigfoot County in 2010.

\item  $P(3) = \frac{120}{1+3.167e^{-0.05(3)}} \approx 32$ Sasquatch.

\item  $t = 20 \ln(3.167) \approx 23$ years.

\item  As $t \rightarrow \infty$, $P(t) \rightarrow 120$.  As time goes by, the Sasquatch Population in Bigfoot County will approach 120.  Graphically,  $y = P(x)$ has a horizontal asymptote $y=120$.

\end{enumerate}}