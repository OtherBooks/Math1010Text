{\label{pHexercise} \index{pH} \index{acidity of a solution ! pH} \index{alkalinity of a solution ! pH} The pH of a solution is a measure of its acidity or alkalinity.  Specifically, $\mbox{pH} = -\log[\mbox{H}^{+}]$ where $[\mbox{H}^{+}]$ is the hydrogen ion concentration in moles per litre.  A solution with a pH less than 7 is an acid, one with a pH greater than 7 is a base (alkaline) and a pH of 7 is regarded as neutral.

\begin{enumerate}

\item The hydrogen ion concentration of pure water is $[\mbox{H}^{+}] = 10^{-7}$.  Find its pH.
\item Find the pH of a solution with $[\mbox{H}^{+}] = 6.3 \times 10^{-13}$.
\item The pH of gastric acid (the acid in your stomach) is about $0.7$.  What is the corresponding hydrogen ion concentration?

\end{enumerate}}
{\begin{enumerate}

\item The pH of pure water is 7.
\item If $[\mbox{H}^{+}] = 6.3 \times 10^{-13}$ then the solution has a pH of 12.2.
\item $[\mbox{H}^{+}] = 10^{-0.7} \approx .1995$ moles per liter.

\end{enumerate}}