{Under optimal conditions, the growth of a certain strain of \textit{E. Coli} is modelled by the Law of Uninhibited Growth $N(t) = N_{\text{\tiny $0$}} e^{kt}$ where $N_{\text{\tiny $0$}}$ is the initial number of bacteria and $t$ is the elapsed time, measured in minutes. From numerous experiments, it has been determined that the doubling time of this organism is 20 minutes. Suppose 1000 bacteria are present initially.

\begin{enumerate}

\item  Find the growth constant $k$. Round your answer to four decimal places.

\item  Find a function which gives the number of bacteria $N(t)$ after $t$ minutes.

\item  How long until there are 9000 bacteria?  Round your answer to the nearest minute.

\end{enumerate}}
{\begin{enumerate} \item  $k = \frac{\ln(2)}{20} \approx 0.0346$

\item  $N(t) = 1000e^{0.0346 t}$

\item  $t = \frac{\ln(9)}{0.0346} \approx 63$ minutes

\end{enumerate}}