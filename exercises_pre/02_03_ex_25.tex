{Assuming no air resistance or forces other than the Earth's gravity, the height above the ground at time $t$ of a falling object is given by $s(t) = -4.9t^{2} + v_{\mbox{\tiny $0$}}t + s_{\mbox{\tiny $0$}}$ where $s$ is in meters, $t$ is in seconds, $v_{\mbox{\tiny $0$}}$ is the object's initial velocity in meters per second and $s_{\mbox{\tiny $0$}}$ is its initial position in meters.  
\label{whatgoesup}

\begin{enumerate}

\item What is the applied domain of this function?
\item Discuss with your classmates what each of $v_{\mbox{\tiny $0$}} > 0, \; v_{\mbox{\tiny $0$}} = 0$ and $v_{\mbox{\tiny $0$}} < 0$ would mean.
\item Come up with a scenario in which $s_{\mbox{\tiny $0$}} < 0$.
\item Let's say a slingshot is used to shoot a marble straight up from the ground $(s_{\mbox{\tiny $0$}} = 0)$ with an initial velocity of 15 meters per second.  What is the marble's maximum height above the ground?  At what time will it hit the ground?
\item Now shoot the marble from the top of a tower which is 25 meters tall.  When does it hit the ground?
\item What would the height function be if instead of shooting the marble up off of the tower, you were to shoot it straight DOWN from the top of the tower?

\end{enumerate}}
{\begin{enumerate}

\item The applied domain is $[0, \infty)$.

\addtocounter{enumii}{2}

\item The height function is this case is $s(t) = -4.9t^{2} + 15t$.  The vertex of this parabola is approximately $(1.53, 11.48)$ so the maximum height reached by the marble is $11.48$ meters.  It hits the ground again when $t \approx 3.06$ seconds.

\item The revised height function is $s(t) = -4.9t^{2} + 15t + 25$ which has zeros at $t \approx -1.20$ and $t \approx 4.26$.  We ignore the negative value and claim that the marble will hit the ground after $4.26$ seconds.

\item Shooting down means the initial velocity is negative so the height functions becomes $s(t) = -4.9t^{2} - 15t + 25$.

\end{enumerate}}