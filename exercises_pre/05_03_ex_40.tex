{\label{pendulumproblem} The period of a pendulum in seconds is given by \[T = 2\pi \sqrt{\dfrac{L}{g}}\](for small displacements) where $L$ is the length of the pendulum in metres and $g = 9.8$ metres per second per second is the acceleration due to gravity.  My Seth-Thomas antique schoolhouse clock needs $T = \frac{1}{2}$ second and I can adjust the length of the pendulum via a small dial on the bottom of the bob.  At what length should I set the pendulum?}
{$9.8 \left(\dfrac{1}{4\pi}\right)^{2} \approx 0.062$ meters or $6.2$ centimetres}