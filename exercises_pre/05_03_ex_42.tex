{According to Einstein's Theory of Special Relativity, the observed mass $m$ of an object is a function of how fast the object is travelling.  Specifically, \[m(x) = \dfrac{m_{r}}{\sqrt{1 - \dfrac{x^{2}}{c^{2}}}}\] where $m(0)=m_{r}$ is the mass of the object at rest, $x$ is the speed of the object and $c$ is the speed of light. 

\begin{enumerate}

\item Find the applied domain of the function.

\item Compute $m(.1c), \, m(.5c), \, m(.9c)$ and $m(.999c)$.

\item As $x \rightarrow c^{-}$, what happens to $m(x)$?

\item How slowly must the object be travelling so that the observed mass is no greater than 100 times its mass at rest?

\end{enumerate}}
{\begin{enumerate}

\item $[0, c)$

\item $~$

\begin{tabular}{ll} 
$m(.1c) = \dfrac{m_{r}}{\sqrt{.99}} \approx 1.005m_{r}$ & $m(.5c) = \dfrac{m_{r}}{\sqrt{.75}} \approx 1.155m_{r}$\\ \smallskip

$m(.9c) = \dfrac{m_{r}}{\sqrt{.19}} \approx 2.294m_{r}$ & $m(.999c) = \dfrac{m_{r}}{\sqrt{.0.001999}} \approx 22.366m_{r}$ \\ \end{tabular}


\item As $x \rightarrow c^{-}, \, m(x) \rightarrow \infty$

\item If the object is travelling no faster than approximately $0.99995$ times the speed of light, then its observed mass will be no greater than $100m_{r}$.

\end{enumerate}}