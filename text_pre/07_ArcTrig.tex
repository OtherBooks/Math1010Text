\section{Inverse Trigonometric Functions}

\label{ArcTrig}

As the title indicates, in this section we concern ourselves with finding inverses of the (circular) trigonometric functions.  Our immediate problem is that, owing to their periodic nature, none of the six circular functions is  one-to-one. To remedy this, we restrict the domains of the circular functions to obtain a one-to-one function.  We first consider $f(x) = \cos(x)$. Choosing the interval $[0,\pi]$ allows us to keep the range as $[-1,1]$ as well as the properties of being smooth and continuous.

\medskip

\noindent\begin{minipage}{\textwidth}
\begin{center}
\myincludegraphics[width=\textwidth]{figures/IntroTrigGraphics/ArcTrig-1}
\end{center} 
\captionsetup{type=figure}
\caption{Restricting the domain of $f(x) = \cos(x)$ to $[0,\pi]$.}
\label{fig:arctrig1}
\end{minipage}

\medskip

Recall from Section \ref{InverseFunctions} that the inverse of a function $f$ is typically denoted $f^{-1}$.  For this reason, some textbooks use the notation $f^{-1}(x) = \cos^{-1}(x)$ for the inverse of $f(x) = \cos(x)$.  The obvious pitfall here is our convention of writing $(\cos(x))^2$ as $\cos^{2}(x)$, $(\cos(x))^3$ as $\cos^{3}(x)$ and so on.  It is far too easy to confuse $\cos^{-1}(x)$ with  $\frac{1}{\cos(x)} = \sec(x)$ so we will not use this notation in our text. (But be aware that many books do! As always, be sure to check the context!) Instead, we use the notation $f^{-1}(x) = \arccos(x)$, read `arc-cosine of $x$'.  To understand the `arc' in `arccosine', recall that an inverse function, by definition, reverses the process of the original function. The function $f(t) = \cos(t)$ takes a real number input $t$, associates it with the angle $\theta = t$ radians, and returns the value $\cos(\theta)$.  Digging deeper,  we have that $\cos(\theta) = \cos(t)$ is the $x$-coordinate of the terminal point on the Unit Circle of an oriented arc of length $|t|$ whose initial point is $(1, 0)$.  Hence, we may view the inputs to $f(t) = \cos(t)$ as oriented arcs and the outputs as $x$-coordinates on the Unit Circle.  The function $f^{-1}$, then, would take $x$-coordinates on the Unit Circle and return oriented arcs, hence the `arc' in arccosine. Figure \ref{fig:arctrig2} shows the graphs of $f(x) = \cos(x)$ and $f^{-1}(x) = \arccos(x)$, where we obtain the latter from the former by reflecting it across the line $y=x$, in accordance with Theorem \ref{inverseuniquegraph}. \index{arccosine ! graph of}

%\mnote{.5}{See page \pageref{wrappingfunction} if you need a review of how we associate real numbers with angles in radian measure.}
\mtable{.4}{Reflecting $y=\cos(x)$ across $y=x$ yields $y=\arccos(x)$}{fig:arctrig2}{
\begin{tabular}{c}
\myincludegraphics[width=0.95\marginparwidth]{figures/IntroTrigGraphics/ArcTrig-2}\\
$f(x) = \cos(x)$, $0 \leq x \leq \pi$\\
\\
\myincludegraphics[width=0.95\marginparwidth]{figures/IntroTrigGraphics/ArcTrig-3}\\
$f^{-1}(x) = \arccos(x)$
\end{tabular}}

We restrict $g(x) = \sin(x)$ in a similar manner, although the interval of choice is $\left[ -\frac{\pi}{2}, \frac{\pi}{2}\right]$.

\medskip

\noindent\begin{minipage}{\textwidth}
\begin{center}
\myincludegraphics[width=\textwidth]{figures/IntroTrigGraphics/ArcTrig-4}
\end{center}
\captionsetup{type=figure}
\caption{Restricting the domain of $f(x) = \sin(x)$ to $\left[-\frac{\pi}{2}, \frac{\pi}{2}\right]$.}
\label{fig:arctrig3}
\end{minipage}

\medskip

It should be no surprise that we call $g^{-1}(x) = \arcsin(x)$, which is read `arc-sine of $x$'. \index{arcsine ! graph of}


We list some important facts about the arccosine and arcsine functions in the following theorem. 

\smallskip

\theorem{arccosinesinefunctionprops}{Properties of the Arccosine and Arcsine Functions}{\index{arccosine ! definition of} \index{arccosine ! properties of} \index{arcsine ! definition of} \index{arcsine ! properties of} 

\begin{itemize}

\item  Properties of $F(x)= \arccos(x)$

\begin{itemize}

\item Domain:  $[-1,1]$

\item Range:  $[0,\pi]$

\item $\arccos(x) = t$ if and only if $0 \leq t \leq \pi$ and $\cos(t) = x$

\item $\cos(\arccos(x)) = x$ provided $-1 \leq x \leq 1$

\item $\arccos(\cos(x)) = x$ provided $0 \leq x \leq \pi$

\end{itemize}

\item  Properties of $G(x) = \arcsin(x)$

\begin{itemize}

\item Domain:  $[-1,1]$

\item Range:  $\left[ -\frac{\pi}{2}, \frac{\pi}{2}\right]$

\item $\arcsin(x) = t$ if and only if $-\frac{\pi}{2} \leq t \leq \frac{\pi}{2}$ and $\sin(t) = x$

\item $\sin(\arcsin(x)) = x$ provided $-1 \leq x \leq 1$

\item $\arcsin(\sin(x)) = x$ provided $-\frac{\pi}{2} \leq x \leq \frac{\pi}{2}$ 

\item additionally, arcsine is odd

\end{itemize}

\end{itemize}
}

\mtable{.7}{Reflecting $y=\sin(x)$ across $y=x$ yields $y=\arcsin(x)$}{fig:arctrig4}{
\begin{tabular}{c}
\myincludegraphics[width=0.95\marginparwidth]{figures/IntroTrigGraphics/ArcTrig-5}\\
$g(x) = \sin(x)$,  $-\frac{\pi}{2} \leq x \leq  \frac{\pi}{2}$\\
\\
\myincludegraphics[width=0.95\marginparwidth]{figures/IntroTrigGraphics/ArcTrig-6}\\
$g^{-1}(x) = \arcsin(x)$
\end{tabular}}


Everything in Theorem \ref{arccosinesinefunctionprops} is a direct consequence of the facts that $f(x) = \cos(x)$ for $0 \leq x \leq \pi$ and $F(x) = \arccos(x)$ are inverses of each other as are $g(x) = \sin(x)$ for $-\frac{\pi}{2} \leq x \leq \frac{\pi}{2}$ and $G(x) = \arcsin(x)$.  It's about time for an example.



\medskip

\example{arccosinesineex}{Evaluating the arcsine and arccosine functions}{

\begin{enumerate}  

\item Find the exact values of the following.

\begin{multicols}{2}

\begin{enumerate}

\item  $\arccos\left(\frac{1}{2}\right)$ 
\item  $\arcsin\left(\frac{\sqrt{2}}{2}\right)$
\item  $\arccos\left(-\frac{\sqrt{2}}{2}\right)$
\item  $\arcsin\left(-\frac{1}{2}\right)$
\item  $\arccos\left( \cos\left(\frac{\pi}{6}\right)\right)$
\item  $\arccos\left( \cos\left(\frac{11\pi}{6}\right)\right)$
\item  $\cos\left(\arccos\left(-\frac{3}{5}\right)\right)$
\item  $\sin\left(\arccos\left(-\frac{3}{5}\right)\right)$

\end{enumerate}

\end{multicols}

\item  \label{algarcsincos} Rewrite the following as algebraic expressions of $x$ and state the domain on which the equivalence is valid.

\begin{multicols}{2}

\begin{enumerate}

\item \label{tanarccos} $\tan\left(\arccos\left(x \right)\right)$

\item  \label{cosarcsin} $\cos\left(2 \arcsin(x)\right)$

\end{enumerate}

\end{multicols}

\end{enumerate}}
{
\begin{enumerate}

\item 

\begin{enumerate}

\item  To find $\arccos\left(\frac{1}{2}\right)$, we need to find the real number $t$ (or, equivalently, an angle measuring $t$ radians) which lies between $0$ and $\pi$ with $\cos(t) = \frac{1}{2}$. We know $t = \frac{\pi}{3}$ meets these criteria, so $\arccos\left(\frac{1}{2}\right)= \frac{\pi}{3}$.

\item  The value of $\arcsin\left(\frac{\sqrt{2}}{2}\right)$ is a real number $t$ between $-\frac{\pi}{2}$ and $\frac{\pi}{2}$ with $\sin(t) = \frac{\sqrt{2}}{2}$.  The number we seek is  $t = \frac{\pi}{4}$. Hence, $\arcsin\left(\frac{\sqrt{2}}{2}\right) = \frac{\pi}{4}$.

\item  The number $t = \arccos\left(-\frac{\sqrt{2}}{2}\right)$ lies in the interval $[0,\pi]$ with $\cos(t) = -\frac{\sqrt{2}}{2}$.  Our answer is $\arccos\left(-\frac{\sqrt{2}}{2}\right) = \frac{3\pi}{4}$.

\item  To find  $\arcsin\left(-\frac{1}{2}\right)$, we seek the number $t$ in the interval $\left[-\frac{\pi}{2}, \frac{\pi}{2}\right]$ with $\sin(t) = -\frac{1}{2}$.  The answer is $t = -\frac{\pi}{6}$ so that $\arcsin\left(-\frac{1}{2}\right) = -\frac{\pi}{6}$.

\item  Since $0 \leq \frac{\pi}{6} \leq \pi$, one option would be to simply invoke Theorem \ref{arccosinesinefunctionprops} to get $\arccos\left( \cos\left(\frac{\pi}{6}\right)\right) = \frac{\pi}{6}$.  However, in order to make sure we understand \textit{why} this is the case, we choose to work the example through using the definition of arccosine.  Working from the inside out,  $\arccos\left( \cos\left(\frac{\pi}{6}\right)\right) = \arccos\left( \frac{\sqrt{3}}{2}\right)$.  Now, $\arccos\left( \frac{\sqrt{3}}{2}\right)$ is the real number $t$ with $0 \leq t \leq \pi$ and $\cos(t) = \frac{\sqrt{3}}{2}$.  We find $t = \frac{\pi}{6}$, so that  $\arccos\left( \cos\left(\frac{\pi}{6}\right)\right) = \frac{\pi}{6}$.

\item Since $\frac{11\pi}{6}$ does not fall between $0$ and $\pi$, Theorem \ref{arccosinesinefunctionprops} does not apply.  We are forced to work through from the inside out starting with  $\arccos\left( \cos\left(\frac{11\pi}{6}\right)\right) = \arccos\left(\frac{\sqrt{3}}{2}\right)$.  From the previous problem, we know $\arccos\left(\frac{\sqrt{3}}{2}\right) = \frac{\pi}{6}$.  Hence,  $\arccos\left( \cos\left(\frac{11\pi}{6}\right)\right) = \frac{\pi}{6}$.

\item  One way to simplify  $\cos\left(\arccos\left(-\frac{3}{5}\right)\right)$ is to use Theorem \ref{arccosinesinefunctionprops} directly.  Since $-\frac{3}{5}$ is between $-1$ and $1$, we have that $\cos\left(\arccos\left(-\frac{3}{5}\right)\right) = -\frac{3}{5}$ and we are done.  However, as before, to really understand \textit{why} this cancellation occurs, we  let $t = \arccos\left(-\frac{3}{5}\right)$.  Then, by definition,  $\cos(t) = -\frac{3}{5}$. Hence, $\cos\left(\arccos\left(-\frac{3}{5}\right)\right) = \cos(t) = -\frac{3}{5}$, and we are finished in (nearly) the same amount of time.

\item  As in the previous example, we let $t = \arccos\left(-\frac{3}{5}\right)$ so that  $\cos(t) = -\frac{3}{5}$ for some $t$ where  $0 \leq t \leq \pi$.  Since $\cos(t) < 0$, we can narrow this down a bit and conclude that $\frac{\pi}{2} < t < \pi$, so that $t$ corresponds to an angle in Quadrant II. In terms of $t$, then, we need to find $\sin\left(\arccos\left(-\frac{3}{5}\right)\right) = \sin(t)$.  Using the Pythagorean Identity $\cos^{2}(t) + \sin^{2}(t) = 1$, we get $\left(-\frac{3}{5}\right)^2 + \sin^{2}(t) = 1$ or $\sin(t) = \pm \frac{4}{5}$.  Since $t$ corresponds to a Quadrants II angle, we choose  $\sin(t) = \frac{4}{5}$.  Hence,  $\sin\left(\arccos\left(-\frac{3}{5}\right)\right) = \frac{4}{5}$.

\end{enumerate}

\drawexampleline

\item

\begin{enumerate}

\mnote{.2}{An alternative approach to finding $\tan(t)$ is to use the identity  $1 + \tan^{2}(t) = \sec^{2}(t)$.  Since $x = \cos(t)$,  $\sec(t) = \frac{1}{\cos(t)} = \frac{1}{x}$.  The reader is invited to work through this approach to see what, if any, difficulties arise.}

\item We begin this problem in the same manner we began the previous two problems.  To help us see the forest for the trees, we let  $t = \arccos(x)$, so our goal is to find a way to express $\tan\left(\arccos\left(x \right)\right) = \tan(t)$ in terms of $x$.  Since $t = \arccos(x)$, we know $\cos(t) = x$ where $0 \leq t \leq \pi$, but since we are after an expression for $\tan(t)$, we know we need to throw out $t = \frac{\pi}{2}$ from consideration.  Hence, either $0 \leq t < \frac{\pi}{2}$ or $\frac{\pi}{2} < t \leq \pi$ so that, geometrically, $t$ corresponds to an angle in Quadrant I or Quadrant II.  One approach to finding $\tan(t)$ is to use the quotient identity $\tan(t) = \frac{\sin(t)}{\cos(t)}$.   Substituting $\cos(t) = x$ into the Pythagorean Identity $\cos^{2}(t) + \sin^{2}(t) = 1$ gives $x^2 + \sin^{2}(t) = 1$, from which we get $\sin(t) = \pm \sqrt{1-x^2}$.  Since $t$ corresponds to angles in Quadrants I and II,   $\sin(t) \geq 0$, so we choose $\sin(t) = \sqrt{1-x^2}$.  Thus, \[\tan(t) = \dfrac{\sin(t)}{\cos(t)} = \dfrac{\sqrt{1-x^2}}{x}\]  To determine the values of $x$ for which this equivalence is valid, we consider our substitution $t = \arccos(x)$.  Since the domain of $\arccos(x)$ is $[-1,1]$, we know we must restrict $-1 \leq x \leq 1$.  Additionally, since we had to discard $t = \frac{\pi}{2}$, we need to discard $x = \cos\left(\frac{\pi}{2}\right) = 0$. Hence, $\tan\left(\arccos\left(x \right)\right) =\frac{\sqrt{1-x^2}}{x}$  is valid for $x$ in $[-1,0)\cup(0,1]$.

\item We proceed as in the previous problem by writing $t = \arcsin(x)$ so that $t$ lies in the interval $\left[ -\frac{\pi}{2}, \frac{\pi}{2}\right]$ with $\sin(t) = x$.  We aim to express $\cos\left(2 \arcsin(x)\right) = \cos(2t)$ in terms of $x$.  Since $\cos(2t)$ is defined everywhere, we get no additional restrictions on $t$ as we did in the previous problem.  We have three choices for rewriting $\cos(2t)$:  $\cos^{2}(t) - \sin^{2}(t)$, $2\cos^{2}(t) - 1$ and $1 - 2\sin^{2}(t)$.  Since we know $x = \sin(t)$, it is easiest to use the last form: \[\cos\left(2 \arcsin(x)\right) = \cos(2t)  = 1 - 2\sin^{2}(t) = 1 - 2x^2\]  To find the restrictions on $x$, we once again appeal to our substitution $t = \arcsin(x)$.  Since $\arcsin(x)$ is defined only for $-1 \leq x \leq 1$, the equivalence $\cos\left(2 \arcsin(x)\right) = 1-2x^2$ is valid only on $[-1,1]$. 

\end{enumerate}

\end{enumerate}
}

\medskip

A few remarks about Example \ref{arccosinesineex} are in order.  Most of the common errors encountered in dealing with the inverse circular functions come from the need to restrict the domains of the original functions so that they are one-to-one.  One instance of this phenomenon is the fact that $\arccos\left( \cos\left(\frac{11\pi}{6}\right)\right) = \frac{\pi}{6}$ as opposed to $\frac{11\pi}{6}$. This is the exact same phenomenon discussed in Section \ref{InverseFunctions} when we saw  $\sqrt{(-2)^2} = 2$ as opposed to $-2$.    Additionally, even though the expression we arrived at in part \ref{cosarcsin} above, namely $1 - 2x^2$, is defined for all real numbers, the equivalence  $\cos\left(2 \arcsin(x)\right) = 1-2x^2$ is valid for only $-1 \leq x \leq 1$.  This is akin to the fact that while the expression $x$ is defined for all real numbers, the equivalence $\left( \sqrt{x} \right)^2 = x$ is valid only for $x \geq 0$.  For this reason, it pays to be careful when we determine the intervals where such equivalences are valid.



\bigskip

The next pair of functions we wish to discuss are the inverses of tangent and cotangent, which are named arctangent and arccotangent, respectively.  First, we restrict $f(x) = \tan(x)$ to its fundamental cycle on $\left(-\frac{\pi}{2}, \frac{\pi}{2}\right)$ to obtain $f^{-1}(x) = \arctan(x)$. Among other things, note that the \textit{vertical} asymptotes $x = -\frac{\pi}{2}$ and $x = \frac{\pi}{2}$ of the graph of $f(x) = \tan(x)$ become the \textit{horizontal} asymptotes $y = -\frac{\pi}{2}$ and $y = \frac{\pi}{2}$ of the graph of $f^{-1}(x) = \arctan(x)$: see Figure \ref{fig:arctrig5}.  \index{arctangent ! graph of}

\mtable{.3}{Reflecting $y=\tan(x)$ across $y=x$ yields $y=\arctan(x)$}{fig:arctrig5}{ 
\begin{tabular}{c}
\myincludegraphics[width=0.6\marginparwidth]{figures/IntroTrigGraphics/ArcTrig-7}\\
$f(x) = \tan(x)$,  $-\frac{\pi}{2} < x <  \frac{\pi}{2}$\\
\\
\myincludegraphics[width=0.95\marginparwidth]{figures/IntroTrigGraphics/ArcTrig-8}\\
$f^{-1}(x) = \arctan(x)$
\end{tabular}}


Next, we restrict $g(x) = \cot(x)$ to its fundamental cycle on $(0,\pi)$ to obtain $g^{-1}(x) = \operatorname{arccot}(x)$.  Once again, the vertical asymptotes $x=0$ and $x=\pi$ of the graph of $g(x) = \cot(x)$ become the horizontal asymptotes $y = 0$ and $y = \pi$ of the graph of $g^{-1}(x) = \operatorname{arccot}(x)$.  We show these graphs in Figure \ref{fig:arctrig5a}; \index{arccotangent ! graph of}
the basic properties of the arctangent and arccotangent functions are given in the following theorem.

\smallskip

\theorem{arctangentcotangentfunctionprops}{Properties of the Arctangent and Arccotangent Functions}{ \index{arctangent ! definition of} \index{arctangent ! properties of} \index{arccotangent ! definition of} \index{arccotangent ! properties of} 

\begin{itemize}

\item Properties of $F(x)= \arctan(x)$

\begin{itemize}

\item Domain: $(-\infty, \infty)$

\item Range: $\left(-\frac{\pi}{2}, \frac{\pi}{2}\right)$

\item  as $x \rightarrow -\infty$, $\arctan(x) \rightarrow -\frac{\pi}{2}^{+}$;  as $x \rightarrow \infty$, $\arctan(x) \rightarrow \frac{\pi}{2}^{-}$

\item  $\arctan(x) = t$ if and only if $-\frac{\pi}{2} < t < \frac{\pi}{2}$ and $\tan(t) = x$

\item  $\arctan(x) = \operatorname{arccot}\left(\frac{1}{x}\right)$ for $x > 0$

\item  $\tan\left(\arctan(x)\right) = x$ for all real numbers $x$

\item  $\arctan(\tan(x)) = x$ provided $-\frac{\pi}{2} < x < \frac{\pi}{2}$

\item additionally, arctangent is odd

\end{itemize}

\item Properties of $G(x) = \operatorname{arccot}(x)$

\begin{itemize}

\item Domain: $(-\infty, \infty)$

\item Range: $(0, \pi)$

\item  as $x \rightarrow -\infty$, $\operatorname{arccot}(x) \rightarrow \pi^{-}$; as $x \rightarrow \infty$, $\operatorname{arccot}(x) \rightarrow 0^{+}$

\item  $\operatorname{arccot}(x) = t$ if and only if $0 <  t < \pi$ and $\cot(t) = x$

\item  $\operatorname{arccot}(x) =\arctan\left(\frac{1}{x}\right)$ for $x > 0$

\item  $\cot\left(\operatorname{arccot}(x)\right) = x$ for all real numbers $x$

\item  $\operatorname{arccot}(\cot(x)) = x$ provided $0 < x < \pi$

\end{itemize}

\end{itemize}
}

\mtable{.7}{Reflecting $y=\cot(x)$ across $y=x$ yields $y=\operatorname{arccot}(x)$}{fig:arctrig5a}{ 
\begin{tabular}{c}
\myincludegraphics[width=0.75\marginparwidth]{figures/IntroTrigGraphics/ArcTrig-9}\\
$g(x) = \cot(x)$,  $0 < x <  \pi$\\
\\
\myincludegraphics[width=0.95\marginparwidth]{figures/IntroTrigGraphics/ArcTrig-10}\\
$g^{-1}(x) = \operatorname{arccot}(x)$
\end{tabular}}

\medskip

\example{ex_arctan}{Evaluating the arctangent and arccotangent functions}{

\begin{enumerate}

\item  Find the exact values of the following.

\begin{multicols}{2}

\begin{enumerate}

\item $\arctan(\sqrt{3})$
\item  $\operatorname{arccot}(-\sqrt{3})$

\setcounter{HW}{\value{enumii}}

\end{enumerate}

\end{multicols}

\begin{multicols}{2}

\begin{enumerate}

\setcounter{enumii}{\value{HW}}

\item  $\cot(\operatorname{arccot}(-5))$
\item  $\sin\left(\arctan\left(-\frac{3}{4}\right)\right)$

\end{enumerate}

\end{multicols}

\item  Rewrite the following as algebraic expressions of $x$ and state the domain on which the equivalence is valid.

\begin{multicols}{2}

\begin{enumerate}

\item  $\tan(2 \arctan(x))$

\item  $\cos(\operatorname{arccot}(2x))$ 

\end{enumerate}

\end{multicols}

\end{enumerate}
}
{
\begin{enumerate}

\item

\begin{enumerate}

\item  We know $\arctan(\sqrt{3})$ is the real number $t$ between $-\frac{\pi}{2}$ and $\frac{\pi}{2}$ with $\tan(t) = \sqrt{3}$.  We find $t = \frac{\pi}{3}$, so $\arctan(\sqrt{3}) = \frac{\pi}{3}$.

\item The real number $t = \operatorname{arccot}(-\sqrt{3})$ lies in the interval $(0,\pi)$ with $\cot(t) = -\sqrt{3}$.  We get $\operatorname{arccot}(-\sqrt{3}) = \frac{5\pi}{6}$.

\item  We can apply Theorem \ref{arctangentcotangentfunctionprops} directly and obtain $\cot(\operatorname{arccot}(-5)) = -5$.  However, working it through provides us with yet another opportunity to understand why this is the case. Letting $t = \operatorname{arccot}(-5)$, we have that $t$ belongs to the interval $(0,\pi)$ and $\cot(t)=-5$.  Hence, $\cot(\operatorname{arccot}(-5)) = \cot(t)=-5$.

\item   We start simplifying  $\sin\left(\arctan\left(-\frac{3}{4}\right)\right)$ by letting $t = \arctan\left(-\frac{3}{4}\right)$.  Then $\tan(t) = -\frac{3}{4}$ for some $-\frac{\pi}{2} < t < \frac{\pi}{2}$.  Since $\tan(t) < 0$, we know, in fact, $-\frac{\pi}{2} < t < 0$.  One way to proceed is to use The Pythagorean Identity, $1 + \cot^{2}(t) = \csc^{2}(t)$, since this relates the reciprocals of $\tan(t)$ and $\sin(t)$ and is valid for all $t$ under consideration.   From  $\tan(t) = -\frac{3}{4}$, we get $\cot(t) = -\frac{4}{3}$.  Substituting, we get $1 + \left(-\frac{4}{3}\right)^2 = \csc^{2}(t)$ so that $\csc(t) = \pm \frac{5}{3}$.  Since $-\frac{\pi}{2} < t < 0$, we choose $\csc(t) = -\frac{5}{3}$, so $\sin(t) = -\frac{3}{5}$. Hence, $\sin\left(\arctan\left(-\frac{3}{4}\right)\right) = -\frac{3}{5}$.

\end{enumerate}

\enlargethispage{2\baselineskip}

\mnote{.5}{It's always a good idea to make sure the identities used in these situations are valid for all values $t$ under consideration.  Check our work back in Example \ref{arccosinesineex}.  Were the identities we used there valid for all $t$ under consideration?  A pedantic point, to be sure, but what else do you expect from this book?}

\item  

\begin{enumerate}

\item If we let $t = \arctan(x)$, then $-\frac{\pi}{2} < t < \frac{\pi}{2}$ and $\tan(t) = x$.   We look for a way to express $\tan(2 \arctan(x)) = \tan(2t)$ in terms of $x$.  Before we get started using identities, we note that $\tan(2t)$ is undefined when $2t = \frac{\pi}{2} + \pi k$ for integers $k$.  Dividing both sides of this equation by $2$ tells us we need to exclude values of $t$ where $t = \frac{\pi}{4} + \frac{\pi}{2} k$, where $k$ is an integer.  The only members of this family which lie in $\left(-\frac{\pi}{2}, \frac{\pi}{2}\right)$ are $t = \pm \frac{\pi}{4}$, which means the values of $t$ under consideration are $\left(-\frac{\pi}{2}, -\frac{\pi}{4}\right) \cup \left(-\frac{\pi}{4}, \frac{\pi}{4}\right) \cup \left(\frac{\pi}{4}, \frac{\pi}{2}\right)$.  Returning to $\arctan(2t)$, we note the double angle identity $\tan(2t) = \frac{2 \tan(t)}{1 - \tan^{2}(t)}$, is valid for all the values of $t$ under consideration, hence we get \[\tan(2 \arctan(x)) = \tan(2t) = \frac{2 \tan(t)}{1 - \tan^{2}(t)}= \frac{2x}{1-x^2}\]  To find where this equivalence is valid we check back with our substitution $t = \arctan(x)$. Since the domain of $\arctan(x)$ is all real numbers, the only exclusions come from the values of $t$ we discarded earlier, $t = \pm \frac{\pi}{4}$.   Since $x =\tan(t)$, this means we exclude $x = \tan\left(\pm \frac{\pi}{4}\right) = \pm 1$.  Hence, the equivalence  $\tan(2 \arctan(x)) =  \frac{2x}{1-x^2}$ holds for all $x$ in  $(-\infty, -1) \cup (-1,1) \cup (1,\infty)$.

\item  To get started, we let $t = \operatorname{arccot}(2x)$ so that  $\cot(t) = 2x$ where $0 < t < \pi$.  In terms of $t$, $\cos(\operatorname{arccot}(2x)) = \cos(t)$, and our goal is to express the latter in terms of $x$.   Since $\cos(t)$ is always defined, there are no additional restrictions on $t$, so we can begin using identities to relate $\cot(t)$ to $\cos(t)$.  The identity $\cot(t) = \frac{\cos(t)}{\sin(t)}$ is valid for $t$ in $(0,\pi)$, so our strategy is to obtain $\sin(t)$ in terms of $x$, then write $\cos(t) = \cot(t) \sin(t)$.   The identity $1 + \cot^{2}(t) = \csc^{2}(t)$ holds for all $t$ in $(0,\pi)$ and relates $\cot(t)$ and $\csc(t) = \frac{1}{\sin(t)}$.  Substituting $\cot(t) =2x$, we get  $1 + (2x)^2 = \csc^{2}(t)$, or $\csc(t) =  \pm \sqrt{4x^2+1}$. Since $t$ is between $0$ and $\pi$, $\csc(t) > 0$, so $\csc(t) =\sqrt{4x^2+1}$ which gives $\sin(t) = \frac{1}{\sqrt{4x^2+1}}$. Hence, \[\cos(\operatorname{arccot}(2x)) = \cos(t) = \cot(t) \sin(t) = \frac{2x}{\sqrt{4x^2+1}}\]   Since $\operatorname{arccot}(2x)$ is defined for all real numbers $x$ and we encountered no additional restrictions on $t$, we have  $\cos\left(\operatorname{arccot}(2x)\right) = \frac{2x}{\sqrt{4x^2+1}}$ for all real numbers $x$. 

\end{enumerate}

\end{enumerate}
}

\pagebreak



The last two functions to invert are secant and cosecant.  A portion of each of their graphs, which were first discussed in Subsection \ref{secantcosecantgraphsection}, are given in Figure \ref{fig:arctrig6} below with the fundamental cycles highlighted. 

\bigskip

\noindent\begin{minipage}{\textwidth}
\begin{center}
\begin{tabular}{cc}
\myincludegraphics[width=0.45\textwidth]{figures/IntroTrigGraphics/ArcTrig-11} & \myincludegraphics[width=0.45\textwidth]{figures/IntroTrigGraphics/ArcTrig-12}\\
The graph $y=\sec(x)$ & The graph $y=\csc(x)$
\end{tabular}
\end{center}
\captionsetup{type=figure}
\caption{The fundamental cycles of $f(x)=\sec(x)$ and $g(x)=\csc(x)$}
\label{fig:arctrig6}
\end{minipage}

\bigskip

It is clear from the graph of secant that we cannot find one single continuous piece of its graph which covers its entire range of $(-\infty, -1] \cup [1, \infty)$ and restricts the domain of the function so that it is one-to-one.  The same is true for cosecant.  Thus in order to define the arcsecant and arccosecant functions, we must settle for a piecewise approach wherein we choose one piece to cover the top of the range, namely  $[1, \infty)$, and another piece to cover the bottom, namely $(-\infty, -1]$.  There are two generally accepted ways make these choices which restrict the domains of these functions so that they are one-to-one.  One approach simplifies the Trigonometry associated with the inverse functions, but complicates the Calculus;  the other makes the Calculus easier, but the Trigonometry less so.  We present both points of view.

%\newpage

\subsection{Inverses of Secant and Cosecant: Trigonometry Friendly Approach}

In this subsection, we restrict the secant and cosecant functions to coincide with the restrictions on cosine and sine, respectively.  For $f(x) = \sec(x)$, we restrict the domain to $\left[0, \frac{\pi}{2}\right) \cup \left( \frac{\pi}{2}, \pi\right]$ (Figure \ref{fig:arctrig7}) \index{arcsecant ! trigonometry friendly ! graph of} and we restrict $g(x) = \csc(x)$ to $\left[-\frac{\pi}{2}, 0\right) \cup \left(0,  \frac{\pi}{2}\right]$ (Figure \ref{fig:arctrig8}. \index{arccosecant ! trigonometry friendly ! graph of}


Note that for both arcsecant and arccosecant, the domain is $(-\infty, -1] \cup [1, \infty)$.  Taking a page from Section \ref{AbsoluteValueFunctions}, we can rewrite this as $\left\{ x : |x| \geq 1\right\}$.  This is often done in Calculus textbooks, so we include it here for completeness.  Using these definitions, we get the following properties of the arcsecant and arccosecant functions.

\smallskip

\theorem{arcsecantcosecantfunctionprops1}{Properties of the Arcsecant and Arccosecant Functions (``Trigonometry Friendly'' version)}{ 

\begin{itemize}

\item Properties of $F(x)= \operatorname{arcsec}(x)$ \index{arcsecant ! trigonometry friendly ! definition of} \index{arcsecant ! trigonometry friendly ! properties of}

\begin{itemize}

\item Domain:  $\left\{ x : |x| \geq 1 \right\} = (-\infty, -1] \cup [1,\infty)$

\item Range:  $\left[0, \frac{\pi}{2} \right) \cup \left(\frac{\pi}{2}, \pi\right]$

\item  as $x \rightarrow -\infty$, $\operatorname{arcsec}(x) \rightarrow \frac{\pi}{2}^{+}$;  as $x \rightarrow \infty$, $\operatorname{arcsec}(x) \rightarrow \frac{\pi}{2}^{-}$

\item  $\operatorname{arcsec}(x) = t$ if and only if $0 \leq t < \frac{\pi}{2}$ or $ \frac{\pi}{2} < t \leq \pi$ and $\sec(t) = x$

\item  $\operatorname{arcsec}(x) = \arccos\left(\frac{1}{x}\right)$ provided $|x| \geq 1$

\item  $\sec\left(\operatorname{arcsec}(x)\right) = x$ provided $|x| \geq 1$

\item  $\operatorname{arcsec}(\sec(x)) = x$ provided $0 \leq x < \frac{\pi}{2}$ or $\frac{\pi}{2} < x \leq \pi$

\end{itemize}

\item  Properties of $G(x) = \operatorname{arccsc}(x)$ \index{arccosecant ! trigonometry friendly ! definition of} \index{arccosecant ! trigonometry friendly ! properties of}

\begin{itemize}

\item  Domain:  $\left\{ x : |x| \geq 1 \right\} = (-\infty, -1] \cup [1,\infty)$

\item Range:  $\left[-\frac{\pi}{2}, 0 \right) \cup \left(0, \frac{\pi}{2} \right]$

\item  as $x \rightarrow -\infty$, $\operatorname{arccsc}(x) \rightarrow 0^{-}$;  as $x \rightarrow \infty$, $\operatorname{arccsc}(x) \rightarrow 0^{+}$

\item  $\operatorname{arccsc}(x) = t$ if and only if $-\frac{\pi}{2} \leq t < 0$ or $0 < t \leq \frac{\pi}{2}$ and $\csc(t) = x$

\item  $\operatorname{arccsc}(x) = \arcsin\left(\frac{1}{x}\right)$ provided $|x| \geq 1$

\item  $\csc\left(\operatorname{arccsc}(x)\right) = x$ provided $|x| \geq 1$

\item  $\operatorname{arccsc}(\csc(x)) = x$ provided $-\frac{\pi}{2} \leq x < 0$ or $0 < x \leq \frac{\pi}{2}$

\item additionally, arccosecant is odd

\end{itemize}

\end{itemize}
}

\medskip

\mtable{.7}{The ``Trigonometry Friendly'' definition of $\operatorname{arcsec}(x)$}{fig:arctrig7}{
\begin{tabular}{c}
\myincludegraphics[width=0.7\marginparwidth]{figures/IntroTrigGraphics/ArcTrig-13}\\
$f(x) = \sec(x)$ on  $\left[0, \frac{\pi}{2}\right) \cup \left( \frac{\pi}{2}, \pi\right]$\\
\\
\myincludegraphics[width=0.9\marginparwidth]{figures/IntroTrigGraphics/ArcTrig-14}\\
$f^{-1}(x)=\operatorname{arcsec}(x)$
\end{tabular}}
 

\mtable{.3}{The ``Trigonometry Friendly'' definition of $\operatorname{arccsc}(x)$}{fig:arctrig8}{
\begin{tabular}{c}
\myincludegraphics[width=0.7\marginparwidth]{figures/IntroTrigGraphics/ArcTrig-15}\\
$g(x) = \csc(x)$ on  $\left[-\frac{\pi}{2}, 0\right) \cup \left(0,  \frac{\pi}{2}\right]$\\
\\
\myincludegraphics[width=0.9\marginparwidth]{figures/IntroTrigGraphics/ArcTrig-16}\\
$g^{-1}(x)=\operatorname{arccsc}(x)$
\end{tabular}}


\example{arcsecantcosecantex1}{Evaluating the arcsecant and arccosecant functions}{

\begin{enumerate}

\item  Find the exact values of the following.

\begin{multicols}{2}

\begin{enumerate}

\item $\operatorname{arcsec}(2)$

\item  $\operatorname{arccsc}(-2)$

\item  $\operatorname{arcsec}\left( \sec\left( \frac{5\pi}{4} \right) \right)$

\item  $\cot\left(\operatorname{arccsc}\left(-3\right)\right)$

\end{enumerate}

\end{multicols}

\item  Rewrite the following as algebraic expressions of $x$ and state the domain on which the equivalence is valid.

\begin{multicols}{2}

\begin{enumerate}

\item  $\tan(\operatorname{arcsec}(x))$

\item  $\cos(\operatorname{arccsc}(4x))$ 

\end{enumerate}

\end{multicols}

\end{enumerate}
}
{
\begin{enumerate}

\item

\begin{enumerate}

\item Using Theorem \ref{arcsecantcosecantfunctionprops1}, we have $\operatorname{arcsec}(2) = \arccos\left(\frac{1}{2}\right) = \frac{\pi}{3}$.

\item  Once again, Theorem \ref{arcsecantcosecantfunctionprops1} gives us $\operatorname{arccsc}(-2) = \arcsin\left(-\frac{1}{2}\right) = -\frac{\pi}{6}$.


\item  Since $\frac{5\pi}{4}$ doesn't fall between $0$ and $\frac{\pi}{2}$ or $\frac{\pi}{2}$ and $\pi$, we cannot use the inverse property stated in Theorem \ref{arcsecantcosecantfunctionprops1}.  We can, nevertheless, begin by working `inside out' which yields  $\operatorname{arcsec}\left( \sec\left( \frac{5\pi}{4} \right) \right) = \operatorname{arcsec}(-\sqrt{2}) = \arccos\left(-\frac{\sqrt{2}}{2}\right) = \frac{3\pi}{4}$.

\item   One way to begin to simplify $\cot\left(\operatorname{arccsc}\left(-3\right)\right)$ is to let $t = \operatorname{arccsc}(-3)$.  Then,  $\csc(t) = -3$ and, since this is negative, we have that $t$ lies in the interval  $\left[ -\frac{\pi}{2},0\right)$.  We are after $\cot\left(\operatorname{arccsc}\left(-3\right)\right) = \cot(t)$, so we use the Pythagorean Identity $1 + \cot^{2}(t) = \csc^{2}(t)$.  Substituting, we have $1 + \cot^{2}(t) = (-3)^2$, or $\cot(t) = \pm \sqrt{8} = \pm 2 \sqrt{2}$.  Since $-\frac{\pi}{2} \leq t < 0$, $\cot(t) < 0$, so we get  $\cot\left(\operatorname{arccsc}\left(-3\right)\right) = -2\sqrt{2}$.

\end{enumerate}

\item 

\begin{enumerate}


\item  We begin simplifying  $\tan(\operatorname{arcsec}(x))$ by letting $t = \operatorname{arcsec}(x)$.  Then, $\sec(t) = x$ for $t$ in $\left[0, \frac{\pi}{2}\right) \cup \left(\frac{\pi}{2}, \pi \right]$, and we seek a formula for $\tan(t)$.  Since $\tan(t)$ is defined for all $t$ values under consideration, we have no additional restrictions on $t$.  To relate $\sec(t)$ to $\tan(t)$, we use the identity $1 + \tan^{2}(t) = \sec^{2}(t)$.  This is valid for all values of $t$ under consideration, and when we substitute $\sec(t) = x$, we get $1 + \tan^{2}(t) = x^2$.  Hence, $\tan(t) = \pm \sqrt{x^2-1}$.  If $t$ belongs to $\left[0, \frac{\pi}{2}\right)$ then $\tan(t) \geq 0$;  if, on the the other hand, $t$ belongs to  $\left(\frac{\pi}{2}, \pi \right]$ then $\tan(t) \leq 0$. As a result, we get a piecewise defined function for $\tan(t)$

\[ 
\tan(t) = \left\{\begin{array}{rr} \sqrt{x^2-1}, & \text{ if } 0 \leq t < \frac{\pi}{2} \\ [5pt] -\sqrt{x^2-1}, & \text{ if } \frac{\pi}{2} < t \leq \pi  \end{array}\right. 
\]

Now we need to determine what these conditions on $t$ mean for $x$.  Since $x = \sec(t)$, when $0 \leq t < \frac{\pi}{2}$, $x \geq 1$, and when $\frac{\pi}{2} < t \leq \pi$, $x \leq -1$.  Since we encountered no further restrictions on $t$, the  equivalence below holds for all $x$ in $(-\infty, -1] \cup [1, \infty)$.

\[ \tan(\operatorname{arcsec}(x)) = \left\{ \begin{array}{rr} \sqrt{x^2-1}, & \text{if $x \geq 1$} \\[5pt]  -\sqrt{x^2-1}, & \text{if $x \leq -1$}  \end{array}\right. \]



\item  To simplify $\cos(\operatorname{arccsc}(4x))$, we start by letting $t = \operatorname{arccsc}(4x)$.   Then $\csc(t) = 4x$ for $t$ in $\left[-\frac{\pi}{2}, 0 \right) \cup \left(0, \frac{\pi}{2}\right]$, and we now set about finding an expression for  $\cos(\operatorname{arccsc}(4x)) = \cos(t)$.  Since $\cos(t)$ is defined for all $t$, we do not encounter any additional restrictions on $t$.  From $\csc(t) = 4x$, we get $\sin(t) = \frac{1}{4x}$, so to find $\cos(t)$, we can make use if the identity $\cos^{2}(t) + \sin^{2}(t) = 1$.  Substituting $\sin(t) = \frac{1}{4x}$  gives $\cos^{2}(t) + \left(\frac{1}{4x}\right)^2 = 1$.  Solving, we get \[\cos(t) = \pm \sqrt{\frac{16x^2-1}{16x^2}} = \pm \frac{\sqrt{16x^2-1}}{4|x|}\]  Since $t$ belongs to $\left[-\frac{\pi}{2}, 0 \right) \cup \left(0, \frac{\pi}{2}\right]$, we know $\cos(t) \geq 0$, so we choose $\cos(t) = \frac{\sqrt{16-x^2}}{4|x|}$. (The absolute values here are necessary, since $x$ could be negative.)  To find the values for which this equivalence is valid, we look back at our original substution,  $t = \operatorname{arccsc}(4x)$.  Since the domain of $\operatorname{arccsc}(x)$ requires its argument $x$ to satisfy $|x| \geq 1$, the domain of $\operatorname{arccsc}(4x)$ requires $|4x| \geq 1$.  We rewrite this inequality and solve to get $x \leq -\frac{1}{4}$ or $x \geq \frac{1}{4}$.  Since we had no additional restrictions on $t$, the equivalence  $\cos(\operatorname{arccsc}(4x)) = \frac{\sqrt{16x^2-1}}{4|x|}$  holds for all $x$ in $\left(-\infty, -\frac{1}{4} \right] \cup \left[\frac{1}{4}, \infty \right)$.  

\end{enumerate}

\end{enumerate}
}

\medskip



\subsection{Inverses of Secant and Cosecant: Calculus Friendly Approach}

In this subsection, we restrict $f(x) = \sec(x)$ to $\left[0, \frac{\pi}{2}\right) \cup \left[\pi, \frac{3\pi}{2}\right)$, \index{arcsecant ! calculus friendly ! graph of} and we restrict $g(x) = \csc(x)$ to $\left(0, \frac{\pi}{2}\right] \cup \left( \pi, \frac{3\pi}{2}\right]$.  \index{arccosecant ! calculus friendly ! graph of}


Using these definitions, we get the following result.

\smallskip

\theorem{arcsecantcosecantfunctionprops2}{Properties of the Arcsecant and Arccosecant Functions (``Calculus Friendly'' version)}{  

\begin{itemize}

\item Properties of $F(x)= \operatorname{arcsec}(x)$ \index{arcsecant ! calculus friendly ! definition of} \index{arcsecant ! calculus friendly ! properties of}

\begin{itemize}

\item Domain: $\left\{ x : |x| \geq 1 \right\} = (-\infty, -1] \cup [1,\infty)$

\item Range: $\left[0, \frac{\pi}{2} \right) \cup \left[\pi, \frac{3\pi}{2} \right)$

\item  as $x \rightarrow -\infty$, $\operatorname{arcsec}(x) \rightarrow \frac{3\pi}{2}^{-}$;  as $x \rightarrow \infty$, $\operatorname{arcsec}(x) \rightarrow \frac{\pi}{2}^{-}$

\item  $\operatorname{arcsec}(x) = t$ if and only if $0 \leq t < \frac{\pi}{2}$ or $ \pi \leq  t < \frac{3\pi}{2}$ and $\sec(t) = x$

\item  $\operatorname{arcsec}(x) = \arccos\left(\frac{1}{x}\right)$ for $x \geq 1$ only (Compare this with the similar result in Theorem \ref{arcsecantcosecantfunctionprops1}.)

\item  $\sec\left(\operatorname{arcsec}(x)\right) = x$ provided $|x| \geq 1$

\item  $\operatorname{arcsec}(\sec(x)) = x$ provided $0 \leq x < \frac{\pi}{2}$ or $ \pi \leq  x < \frac{3\pi}{2}$


\end{itemize}

\item Properties of $G(x) = \operatorname{arccsc}(x)$ \index{arccosecant ! calculus friendly ! definition of} \index{arccosecant ! calculus friendly ! properties of}

\begin{itemize}

\item Domain: $\left\{ x : |x| \geq 1 \right\} = (-\infty, -1] \cup [1,\infty)$

\item Range: $\left(0, \frac{\pi}{2} \right] \cup \left( \pi, \frac{3\pi}{2} \right]$

\item  as $x \rightarrow -\infty$, $\operatorname{arccsc}(x) \rightarrow \pi^{+}$;  as $x \rightarrow \infty$, $\operatorname{arccsc}(x) \rightarrow 0^{+}$

\item  $\operatorname{arccsc}(x) = t$ if and only if $0 < t \leq \frac{\pi}{2}$ or $\pi < t \leq \frac{3\pi}{2}$ and $\csc(t) = x$

\item  $\operatorname{arccsc}(x) = \arcsin\left(\frac{1}{x}\right)$ for $x \geq 1$ only (Compare this with the similar result in Theorem \ref{arcsecantcosecantfunctionprops1}.)

\item  $\csc\left(\operatorname{arccsc}(x)\right) = x$ provided $|x| \geq 1$

\item  $\operatorname{arccsc}(\csc(x)) = x$ provided $0 < x \leq \frac{\pi}{2}$ or $\pi < x \leq \frac{3\pi}{2}$

\end{itemize}

\end{itemize}
}

\medskip


\mtable{.71}{The ``Calculus Friendly'' definition of $\operatorname{arcsec}(x)$}{fig:arctrig9}{
\begin{tabular}{c}
\myincludegraphics[width=0.8\marginparwidth]{figures/IntroTrigGraphics/ArcTrig-17}\\
$f(x) = \sec(x)$ on  $\left[0, \frac{\pi}{2}\right) \cup \left[ \pi, \frac{3\pi}{2}\right)$\\
\\
\myincludegraphics[width=0.9\marginparwidth]{figures/IntroTrigGraphics/ArcTrig-18}\\
$f^{-1}(x)=\operatorname{arcsec}(x)$
\end{tabular}}
 

\mtable{.28}{The ``Calculus Friendly definition of $\operatorname{arccsc}(x)$}{fig:arctrig10}{
\begin{tabular}{c}
\myincludegraphics[width=0.8\marginparwidth]{figures/IntroTrigGraphics/ArcTrig-19}\\
$g(x) = \csc(x)$ on  $\left(0, \frac{\pi}{2}\right] \cup \left(\pi,  \frac{3\pi}{2}\right]$\\
\\
\myincludegraphics[width=0.9\marginparwidth]{figures/IntroTrigGraphics/ArcTrig-20}\\
$g^{-1}(x)=\operatorname{arccsc}(x)$
\end{tabular}}

Our next example is a duplicate of Example \ref{arcsecantcosecantex1}.  The interested reader is invited to compare and contrast the solution to each.

\medskip

\example{arcsecantcosecantex2}{Evaluating the arcsecant and arccosecant functions}{

\begin{enumerate}

\item  Find the exact values of the following.

\begin{multicols}{2}

\begin{enumerate}

\item $\operatorname{arcsec}(2)$

\item  $\operatorname{arccsc}(-2)$

\item  $\operatorname{arcsec}\left( \sec\left( \frac{5\pi}{4} \right) \right)$

\item  $\cot\left(\operatorname{arccsc}\left(-3\right)\right)$

\end{enumerate}

\end{multicols}

%\enlargethispage{.25in}

\item  Rewrite the following as algebraic expressions of $x$ and state the domain on which the equivalence is valid.

\begin{multicols}{2}

\begin{enumerate}

\item  $\tan(\operatorname{arcsec}(x))$

\item  $\cos(\operatorname{arccsc}(4x))$ 

\end{enumerate}

\end{multicols}

\end{enumerate}
}
{
\begin{enumerate}
\item
\begin{enumerate}

\item  Since $2 \geq 1$, we can use Theorem \ref{arcsecantcosecantfunctionprops2} to get $\operatorname{arcsec}(2) = \arccos\left(\frac{1}{2}\right) = \frac{\pi}{3}$.

\item  Unfortunately, $-2$ is not greater to or equal to $1$, so we cannot apply Theorem \ref{arcsecantcosecantfunctionprops2} to  $\operatorname{arccsc}(-2)$ and convert this into an arcsine problem.  Instead, we appeal to the definition.  The real number $t = \operatorname{arccsc}(-2)$ lies in $\left(0,\frac{\pi}{2} \right] \cup \left(\pi, \frac{3\pi}{2}\right]$ and satisfies $\csc(t) = -2$.  The $t$ we're after is $t = \frac{7\pi}{6}$, so $\operatorname{arccsc}(-2) = \frac{7\pi}{6}$.

\item  Since $\frac{5\pi}{4}$ lies between $\pi$ and $\frac{3\pi}{2}$, we may apply Theorem \ref{arcsecantcosecantfunctionprops2} directly to simplify $\operatorname{arcsec}\left( \sec\left( \frac{5\pi}{4} \right) \right) = \frac{5\pi}{4}$.  We encourage the reader to work this through using the definition as we have done in the  previous examples to see how it goes.

\item  To help simplify $\cot\left(\operatorname{arccsc}\left(-3\right)\right)$ we define $t = \operatorname{arccsc}\left(-3\right)$ so that  $\cot\left(\operatorname{arccsc}\left(-3\right)\right) = \cot(t)$.  We know $\csc(t) = -3$, and since this is negative,  $t$ lies in $\left( \pi, \frac{3\pi}{2}\right]$.  Using the identity $1 + \cot^{2}(t) = \csc^{2}(t)$, we find $1 + \cot^{2}(t) = (-3)^2$ so that $\cot(t) = \pm \sqrt{8} = \pm 2 \sqrt{2}$.  Since $t$ is in the interval $\left(\pi, \frac{3\pi}{2}\right]$, we know $\cot(t) > 0$.  Our answer is $\cot\left(\operatorname{arccsc}\left(-3\right)\right) = 2 \sqrt{2}$.

\end{enumerate}

\drawexampleline

\item 
\begin{enumerate}

\item  We begin simplifying  $\tan(\operatorname{arcsec}(x))$ by letting $t = \operatorname{arcsec}(x)$.  Then, $\sec(t) = x$ for $t$ in $\left[0, \frac{\pi}{2} \right) \cup \left[\pi, \frac{3\pi}{2} \right)$, and we seek a formula for $\tan(t)$.  Since $\tan(t)$ is defined for all $t$ values under consideration, we have no additional restrictions on $t$.  To relate $\sec(t)$ to $\tan(t)$, we use the identity $1 + \tan^{2}(t) = \sec^{2}(t)$.  This is valid for all values of $t$ under consideration, and when we substitute $\sec(t) = x$, we get $1 + \tan^{2}(t) = x^2$.  Hence, $\tan(t) = \pm \sqrt{x^2-1}$. Since $t$ lies in $\left[0, \frac{\pi}{2} \right) \cup \left[\pi, \frac{3\pi}{2} \right)$, $\tan(t) \geq 0$, so we choose $\tan(t) = \sqrt{x^2-1}$.  Since we found no additional restrictions on $t$, the equivalence $\tan(\operatorname{arcsec}(x)) = \sqrt{x^2-1}$ holds for all $x$ in the domain of $t = \operatorname{arcsec}(x)$, namely $(-\infty, -1] \cup [1,\infty)$.

\item To simplify $\cos(\operatorname{arccsc}(4x))$, we start by letting $t = \operatorname{arccsc}(4x)$.   Then $\csc(t) = 4x$ for $t$ in $\left(0, \frac{\pi}{2} \right] \cup \left(\pi, \frac{3\pi}{2} \right]$, and we now set about finding an expression for  $\cos(\operatorname{arccsc}(4x)) = \cos(t)$.  Since $\cos(t)$ is defined for all $t$, we do not encounter any additional restrictions on $t$.  From $\csc(t) = 4x$, we get $\sin(t) = \frac{1}{4x}$, so to find $\cos(t)$, we can make use if the identity $\cos^{2}(t) + \sin^{2}(t) = 1$.  Substituting $\sin(t) = \frac{1}{4x}$  gives $\cos^{2}(t) + \left(\frac{1}{4x}\right)^2 = 1$.  Solving, we get \[\cos(t) = \pm \sqrt{\frac{16x^2-1}{16x^2}} = \pm \frac{\sqrt{16x^2-1}}{4|x|}\]  If $t$ lies in $\left(0, \frac{\pi}{2} \right]$, then $\cos(t) \geq 0$, and we choose  $\cos(t) = \frac{\sqrt{16x^2-1}}{4|x|}$. Otherwise, $t$ belongs to $\left( \pi, \frac{3\pi}{2} \right]$ in which case $\cos(t) \leq 0$, so, we choose $\cos(t) = -\frac{\sqrt{16x^2-1}}{4|x|}$ This leads us to a (momentarily) piecewise defined function for $\cos(t)$

\[ 
\cos(t) = \left\{ \begin{array}{rr}
			 \dfrac{\sqrt{16x^2-1}}{4|x|}, & \text{if $0 \leq t \leq \frac{\pi}{2}$} \\ [10pt]
			   -\dfrac{\sqrt{16x^2-1}}{4|x|}, & \text{if $\pi < t \leq \frac{3\pi}{2}$}  
			    \end{array}\right. 
\]

We now see what these restrictions mean in terms of $x$.  Since $4x = \csc(t)$, we get that for $0 \leq t \leq \frac{\pi}{2}$, $4x \geq 1$, or $x \geq \frac{1}{4}$.  In this case, we can simplify $|x| = x$ so 
\[
\cos(t) = \frac{\sqrt{16x^2-1}}{4|x|} = \frac{\sqrt{16x^2-1}}{4x}
\]  
Similarly, for $\pi < t \leq \frac{3\pi}{2}$, we get $4x \leq -1$, or $x \leq -\frac{1}{4}$.  In this case, $|x| = -x$, so we also get 
\[
\cos(t) = -\frac{\sqrt{16x^2-1}}{4|x|}  = -\frac{\sqrt{16x^2-1}}{4(-x)} =  \frac{\sqrt{16x^2-1}}{4x}
\]  
Hence, in all cases, $\cos(\operatorname{arccsc}(4x)) =  \frac{\sqrt{16x^2-1}}{4x}$, and this equivalence is valid for all $x$ in the domain of $t = \operatorname{arccsc}(4x)$, namely \\$\left(-\infty, -\frac{1}{4}\right] \cup \left[ \frac{1}{4}, \infty \right)$  

\end{enumerate}

\end{enumerate}
}





\printexercises{exercises_pre/07_06_exercises}