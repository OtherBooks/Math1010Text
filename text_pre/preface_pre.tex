\thispagestyle{empty}
\Huge
\noindent {\bf \textsc{Preface}}\\
\normalsize

One of the challenges with a new course like Math 1010 is finding a suitable textbook for the course. This is made additionally difficult for a course that covers two topics -- Precalculus and Calculus -- that are usually offered as separate courses, with separate texts. I reviewed a number of commercially available options, but these all had two things in common: they did not quite meet our needs, and they were all very expensive (some were as much as \$400).

Since writing a new textbook from scratch is a huge undertaking, requiring resources (like time) we simply did not have, I chose to explore non-commercial options. This took a bit of searching, since non-commercial texts, while inexpensive (or free), are of varying quality. Fortunately, there are some decent texts out there. Unfortunately, I couldn't find a single text that covered all of the material we need for Math 1010.

To get around this problem, I have selected two textbooks as our primary sources for the course. The first is \textit{Precalculus}, version 3, by Carl Stitz and Jeff Zeager. The second is \textit{APEX Calculus I},  version 3.0, by Hartman et al. Both texts have two very useful advantages. First, they're both free (as in beer): you can download either text in PDF format from the authors' web pages. Second, they're also \textit{open source} texts (that is, free, as in speech). Both books are written using the \LaTeX markup language, as is typical in mathematics publishing. What is not typical is that the authors of both texts make their source code freely available, allowing others (such as myself) to edit and customize the books as they see fit.

In the first iteration of this project (Fall 2015), I was only able to edit each text individually for length and content, resulting in two separate textbooks for Math 1010. This time around, I've had enough time to take the content of the Precalculus textbook and adapt its source code to be compatible with the formatting of the Calculus textbook, allowing me to produce a single textbook for all of Math 1010.

The ``Complete (and Current) Edition'' represents the most up to date version of the text, with all possible sections included. There is more material here than an instructor can reasonably expect to cover in one semester. The book can (and will) be abridged and customized for each particular offering of Math 1010. 

The book is very much a work in progress, and I will be editing it regularly. Feedback is always welcome. 

\newpage

\noindent\textbf{\large Acknowledgements}\\

First and foremost, I need to thank the authors of the two textbooks that provide the source material for this text. Without their hard work, and willingness to make their books (and the source code) freely available, it would not have been possible to create an affordable textbook for this course. You can find the original textbooks at their websites:

\bigskip


\href{http://www.stitz-zeager.com}{www.stitz-zeager.com}, for the \textit{Precalculus} textbook, by Stitz and Zeager, and

\bigskip


\href{http://www.apexcalculus.com}{apexcalculus.com}, for the \apex\ \textit{Calculus} textbook, by Hartman et al.

\bigskip

I'd also like to thank Dave Morris for help with converting the graphics in the Precalculus textbook to work with the formatting code of the APEX text, Howard Cheng for providing some C++ code to convert the exercises, and the other faculty members involved with this course --- Alia Hamieh, David Kaminsky, and Nicole Wilson --- for their input on the content of the text.

\vspace{1in}

\begin{raggedright}
Sean Fitzpatrick\\
Department of Mathematics and Computer Science\\
University of Lethbridge\\
June, 2016
\end{raggedright}




