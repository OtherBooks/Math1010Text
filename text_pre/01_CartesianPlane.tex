\section{The Cartesian Coordinate Plane}
\label{CartesianPlane}

In order to visualize the pure excitement that is Precalculus, we need to unite Algebra and Geometry.  Simply put, we must find a way to draw algebraic things.  Let's start with possibly the greatest mathematical achievement of all time: the \index{Cartesian coordinate plane} \sword{Cartesian Coordinate Plane}.
Imagine two real number lines crossing at a right angle at $0$ as drawn below.

\mnote{.85}{The Cartesian Plane is named in honour of \href{http://en.wikipedia.org/wiki/Descartes}{\underline{Ren\'{e} Descartes}}.} 

\begin{center}

\myincludegraphics{figures/RelationsandFunctionsGraphics/CartesianPlane-21}

\end{center}

\medskip

The horizontal number line is usually called the \index{$x$-axis} \sword{\boldmath $x$-axis} while the vertical number line is usually called the \index{$y$-axis} \sword{\boldmath $y$-axis}.  As with the usual number line, we imagine these axes extending off indefinitely in both directions.   Having two number lines allows us to locate the positions of points off of the number lines as well as points on the lines themselves.  

\mnote{.4}{Usually extending off  towards infinity is indicated by arrows, but here, the arrows are used to indicate the \textit{direction} of increasing values of $x$ and $y$.}


\mnote{.2}{The names of the coordinates can vary depending on the context of the application.  If, for example, the horizontal axis represented time we might choose to call it the $t$-axis.  The first number in the ordered pair would then be the $t$-coordinate.}

\medskip

For example, consider the point $P$ on the next page.  To use the numbers on the axes to label this point, we imagine dropping a vertical line from the $x$-axis to $P$ and extending a horizontal line from the $y$-axis to $P$.  This process is sometimes called `projecting' the point $P$ to the $x$- (respectively $y$-) axis.  We then describe the point $P$ using the \index{ordered pair} \sword{ordered pair} $(2,-4)$.  The first number in the ordered pair is called the \index{abscissa} \sword{abscissa} or \index{$x$-coordinate} \sword{\boldmath $x$-coordinate} and the second is called the \index{ordinate} \sword{ordinate} or \index{$y$-coordinate} \sword{\boldmath $y$-coordinate}.  Taken together, the ordered pair $(2,-4)$ comprise the \index{coordinates ! Cartesian}\index{Cartesian coordinates}\sword{Cartesian coordinates} of the point $P$. In practice, the distinction between a point and its coordinates is blurred; for example, we often speak of `the point $(2,-4)$.'  We can think of $(2,-4)$ as instructions on how to reach $P$ from the \index{origin} {\bf origin} $(0, 0)$ by moving $2$ units to the right and $4$ units downwards.  Notice that the order in the \underline{ordered} pair is important $-$ if we wish to plot the point $(-4,2)$, we would move to the left $4$ units from the origin and then move upwards $2$ units, as below on the right.


\medskip

\noindent\ifthenelse{\isodd{\thepage}}{\hskip-110pt}{}
\noindent\begin{minipage}{1.3\linewidth}
\begin{tabular}{cc}
\myincludegraphics{figures/RelationsandFunctionsGraphics/CartesianPlane-22} &

\myincludegraphics{figures/RelationsandFunctionsGraphics/CartesianPlane-23}\\

\end{tabular}
\end{minipage}

When we speak of the Cartesian Coordinate Plane, we mean the set of all possible ordered pairs $(x,y)$ as $x$ and $y$ take values from the real numbers.  Below is a summary of important facts about Cartesian coordinates.

\mnote{.5}{Cartesian coordinates are sometimes referred to as \textit{rectangular coordinates}, to distinguish them from other coordinate systems such as \textit{polar coordinates}. }


\smallskip 

\keyidea{idea:cartfacts}{Important Facts about the Cartesian Coordinate Plane}{
\begin{itemize}

\item $(a,b)$ and $(c,d)$ represent the same point in the plane if and only if $a = c$ and $b = d$.

\item  $(x,y)$ lies on the $x$-axis if and only if $y = 0$.

\item  $(x,y)$ lies on the $y$-axis if and only if $x=0$.

\item The origin is the point $(0,0)$.  It is the only point common to both axes.

%\smallskip

\end{itemize}
}

\mnote{0.3}{The letter $O$ is almost always reserved for the origin.}
\example{ex_cartplot}{Plotting points in the Cartesian Plane}{
Plot the following points: $A(5,8)$, $B\left(-\frac{5}{2}, 3\right)$, $C(-5.8, -3)$, $D(4.5, -1)$, $E(5,0)$, $F(0,5)$, $G(-7,0)$, $H(0, -9)$, $O(0,0)$.
}
{
To plot these points, we start at the origin and move to the right if the $x$-coordinate is positive; to the left if it is negative.   Next, we move up if the $y$-coordinate is positive or down if it is negative.  If the $x$-coordinate is $0$, we start at the origin and move along the $y$-axis only.  If the  $y$-coordinate is $0$ we move along the $x$-axis only.


\begin{center}

\myincludegraphics{figures/RelationsandFunctionsGraphics/CartesianPlane-24}

\end{center}
}

\medskip

%\pagebreak

The axes divide the plane into four regions called \index{quadrants} \sword{quadrants}.  They are labelled with Roman numerals and proceed counterclockwise around the plane: see Figure \ref{quadrant}.


\mtable{.35}{The four quadrants of the Cartesian plane}{quadrant}{
\begin{tabular}{c}
\myincludegraphics[scale=0.9]{figures/RelationsandFunctionsGraphics/CartesianPlane-25}
\end{tabular}
}

For example, $(1,2)$ lies in Quadrant I, $(-1,2)$ in Quadrant II, $(-1,-2)$ in Quadrant III and $(1,-2)$ in Quadrant IV.  If a point other than the origin happens to lie on the axes, we typically refer to that point as lying on the positive or negative $x$-axis (if $y = 0$) or on the positive or negative $y$-axis (if $x = 0$).  For example, $(0,4)$ lies on the positive $y$-axis whereas $(-117,0)$ lies on the negative $x$-axis.  Such points do not belong to any of the four quadrants.

\smallskip

One of the most important concepts in all of Mathematics is \textbf{symmetry}. There are many types of symmetry in Mathematics, but three of them can be discussed easily using Cartesian Coordinates.

\medskip

\definition{symmetrydefn}{Symmetry in the Cartesian Plane}{
Two points $(a,b)$ and $(c,d)$ in the plane are said to be

\begin{itemize}

\item \index{symmetry ! about the $x$-axis} \sword{symmetric about the \boldmath $x$-axis} if $a = c$ and $b = -d$

\item \index{symmetry ! about the $y$-axis} \sword{symmetric about the \boldmath $y$-axis} if $a = -c$ and $b = d$

\item \index{symmetry ! about the origin} \sword{symmetric about the origin} if $a = -c$ and $b = -d$

\end{itemize}
}

\pagebreak



In Figure \ref{fig:plansymm}, $P$ and $S$ are symmetric about the $x$-axis, as are $Q$ and $R$;  $P$ and $Q$ are symmetric about the $y$-axis, as are $R$ and $S$;  and $P$ and $R$ are symmetric about the origin, as are $Q$ and $S$.

\mtable{.8}{The three types of symmetry in the plane}{fig:plansymm}{
\begin{tabular}{c}
\myincludegraphics{figures/RelationsandFunctionsGraphics/CartesianPlane-26}
\end{tabular}
}

\medskip

\example{ex_planesymm}{Finding points exhibiting symmetry}{
Let $P$ be the point $(-2,3)$.  Find the points which are symmetric to $P$ about the:

\begin{multicols}{3}

\begin{enumerate}

\item  $x$-axis

\item  $y$-axis

\item  origin

\end{enumerate}

\end{multicols}

Check your answer by plotting the points.
}
{
The figure after Definition \ref{symmetrydefn} gives us a good way to think about finding symmetric points in terms of taking the opposites of the $x$- and/or $y$-coordinates of $P(-2,3)$.

\begin{enumerate}

\item  To find the point symmetric about the $x$-axis, we replace the $y$-coordinate with its opposite to get  $(-2,-3)$.

\item  To find the point symmetric about the $y$-axis, we replace the $x$-coordinate with its opposite to get $(2,3)$.

\item  To find the point symmetric about the origin, we replace the $x$- and $y$-coordinates with their opposites to get $(2,-3)$.

The points are plotted in Figure \ref{fig:pointsymm}.

\end{enumerate}

\mfigure{.5}{The point $P(-2,3)$ and its three reflections}{fig:pointsymm}{figures/RelationsandFunctionsGraphics/CartesianPlane-27}


}


\medskip

One way to visualize the processes in the previous example is with the concept of a \index{reflection ! of a point} \sword{reflection}.  If we start with our point $(-2,3)$ and pretend that the $x$-axis is a mirror, then the reflection of $(-2,3)$ across the $x$-axis would lie at $(-2,-3)$.  If we pretend that the $y$-axis is a mirror, the reflection of $(-2,3)$ across that axis would be $(2,3)$.  If we reflect across the $x$-axis and then the $y$-axis, we would go from $(-2,3)$ to $(-2,-3)$ then to $(2,-3)$, and so we would end up at the point symmetric to $(-2,3)$ about the origin.  We summarize and generalize this process below.

\medskip

\keyidea{idea:planereflect}{Reflections in the Cartesian Plane}{
To reflect a point $(x,y)$ about the:

\begin{itemize}

\item  $x$-axis, replace $y$ with $-y$.

\item  $y$-axis, replace $x$ with $-x$.

\item  origin, replace $x$ with $-x$ and $y$ with $-y$.

\end{itemize}
}

\subsection{Distance in the Plane}

Another important concept in Geometry is the notion of length.  If we are going to unite Algebra and Geometry using the Cartesian Plane, then we need to develop an algebraic understanding of what distance in the plane means.  Suppose we have two points, $P\left(x_{0}, y_{0}\right)$ and $Q\left(x_{1}, y_{1}\right),$ in the plane. By the \index{distance ! definition} \sword{distance} $d$  between $P$ and $Q$, we mean the length of the line segment joining $P$ with $Q$.  (Remember, given any two distinct points in the plane, there is a unique line containing both points.)  Our goal now is to create an algebraic formula to compute the distance between these two points. Consider the generic situation in Figure \ref{fig:cartdist}.

\medskip

\mtable{.7}{Distance between $P$ and $Q$}{fig:cartdist}{
\begin{tabular}{c}
\myincludegraphics{figures/RelationsandFunctionsGraphics/CartesianPlane-28} \\
\\
\myincludegraphics{figures/RelationsandFunctionsGraphics/CartesianPlane-29}
\end{tabular}
}

\medskip

With a little more imagination, we can envision a right triangle whose hypotenuse has length $d$ as drawn above on the right.  From the latter figure, we see that the lengths of the legs of the triangle are $\left|x_{1} - x_{0}\right|$ and $\left|y_{1} - y_{0}\right|$ so the \href{http://en.wikipedia.org/wiki/Pythagorean_Theorem}{\underline{Pythagorean Theorem}} gives us
 
 \[ \left|x_{1} - x_{0}\right|^2 + \left|y_{1} - y_{0}\right|^2 = d^2\]
 \[ \left(x_{1} - x_{0}\right)^2 + \left(y_{1} - y_{0}\right)^2 = d^2\]
 
(Do you remember why we can replace the absolute value notation with parentheses?)  By extracting the square root of both sides of the second equation and using the fact that distance is never negative, we get
 
\medskip
 
\keyidea{distanceformula}{The Distance Formula}{
\index{distance ! distance formula}  The distance $d$ between the points $P\left(x_{0}, y_{0}\right)$ and $Q\left(x_{1}, y_{1}\right)$ is:
 
\[d = \sqrt{ \left(x_{1} - x_{0}\right)^2 + \left(y_{1} - y_{0}\right)^2} \]
}

\medskip

It is not always the case that the points $P$ and $Q$ lend themselves to constructing such a triangle.  If the points $P$ and $Q$ are arranged vertically or horizontally, or describe the exact same point, we cannot use the above geometric argument to derive the distance formula.  It is left to the reader in Exercise \ref{distanceothercases} to verify Equation \ref{distanceformula} for these cases.

\medskip

\example{ex_twoptdist}{Distance between two points}{
Find and simplify the distance between $P(-2,3)$ and  $Q(1,-3)$.  
}
{
\begin{align*}
 d & =  \sqrt{\left(x_{1} - x_{0} \right)^2 + \left(y_{1} - y_{0} \right)^2} \\
   & =  \sqrt{ (1-(-2))^2 + (-3-3)^2} \\
   & =  \sqrt{9 + 36} \\
   & =  3 \sqrt{5}
\end{align*}

So the distance is $3 \sqrt{5}$. 
}

\medskip

\example{ex_pointwithdist}{Finding points at a given distance}{
Find all of the points with $x$-coordinate $1$ which are $4$ units from the point $(3,2)$.
}
{
We shall soon see that the points we wish to find are on the line $x=1$, but for now we'll just view them as points of the form $(1,y)$.  

We require that the distance from $(3,2)$ to $(1,y)$ be $4$.  The Distance Formula, Equation \ref{distanceformula}, yields

\begin{align*} 
d &  =  \sqrt{\left(x_{1}-x_{0}\right)^2+\left(y_{1}-y_{0}\right)^2} \\
4 &  =  \sqrt{(1-3)^2+(y-2)^2}  \\
4  & =  \sqrt{4+(y-2)^2}  \\ 
4^2 & =  \left(\sqrt{4+(y-2)^2}\right)^2   \tag*{squaring both sides} \\
16 & =  4+(y-2)^2 \\
12 & =  (y-2)^2  \\
(y-2)^2 & =  12   \\
y - 2 & =  \pm \sqrt{12} \tag*{extracting the square root} \\
y-2 & =  \pm 2 \sqrt{3}  \\
y & =  2 \pm 2 \sqrt{3}   
\end{align*}


We obtain two answers:  $(1, 2 + 2 \sqrt{3})$ and $(1, 2-2 \sqrt{3}).$  The reader is encouraged to think about why there are two answers.}

\mtable{.8}{Diagram for Example \ref{ex_pointwithdist}}{fig:pointdist}{
\begin{tabular}{c}
\myincludegraphics{figures/RelationsandFunctionsGraphics/CartesianPlane-30}
\end{tabular}
}

\medskip

Related to finding the distance between two points is the problem of finding the \index{midpoint ! definition of} \sword{midpoint} of the line segment connecting two points.  Given two points, $P\left(x_{0}, y_{0}\right)$ and $Q\left(x_{1}, y_{1}\right)$, the \textbf{midpoint} $M$  of $P$ and $Q$ is defined to be the point on the line segment connecting $P$ and $Q$ whose distance from $P$ is equal to its distance from  $Q$.  

\mtable{.4}{The midpoint of a line segment}{fig:midptseg}{
\begin{tabular}{c}
\myincludegraphics{figures/RelationsandFunctionsGraphics/CartesianPlane-31}
\end{tabular}
}

\medskip

\keyidea{midpointformula}{The Midpoint Formula}{
\index{midpoint ! midpoint formula}The midpoint $M$ of the line segment connecting $P\left(x_{0}, y_{0}\right)$ and $Q\left(x_{1}, y_{1}\right)$ is:

\[ 
M = \left( \dfrac{x_{0} + x_{1}}{2} , \dfrac{y_{0} + y_{1}}{2} \right)
\]
}

\medskip

If we let $d$ denote the distance between $P$ and $Q$, we leave it as Exercise \ref{verifymidpointformula} to show that the distance between $P$ and $M$ is $d/2$ which is the same as the distance between $M$ and $Q$.  This suffices to show that Key Idea \ref{midpointformula} gives the coordinates of the midpoint.

\medskip

\example{ex_findmidpoint}{Finding the midpoint of a line segment}
{
Find the midpoint of the line segment connecting $P(-2,3)$ and  $Q(1,-3)$.  
}
{

\begin{align*}
 M & = \left( \dfrac{x_{0}+x_{1}}{2},  \dfrac{y_{0}+y_{1}}{2} \right) \\
   & = \left( \dfrac{(-2)+1}{2},  \dfrac{3+(-3)}{2} \right)  = \left(- \dfrac{1}{2}, \dfrac{0}{2} \right) \\
   & = \left(- \dfrac{1}{2}, 0 \right) 
\end{align*}   
The midpoint is  $\left(- \frac{1}{2}, 0 \right)$.}

%\closegraphsfile
\printexercises{exercises_pre/01_01_exercises}
