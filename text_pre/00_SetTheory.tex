

\section{Some Basic Set Theory Notions}
\label{SetTheory}

%\label{SetTheoryNotions}


%\label{SetsofNumbers}
While the authors would like nothing more than to delve quickly and deeply into the sheer excitement that is \textit{Precalculus}, experience has taught us that a brief refresher on some basic notions is welcome, if not completely necessary, at this stage.  To that end, we present a brief summary of `set theory' and some of the associated vocabulary and notations we use in the text. Like all good Math books, we begin with a definition.


\definition{setdef}{Set}{
A \sword{set}\index{set ! definition of} is a well-defined collection of objects which are called the `elements' of the set.  Here, `well-defined' means that it is possible to determine if something belongs to the collection or not, without prejudice. 
}

\smallskip

For example, the collection of letters that make up the word ``pronghorns'' is well-defined and is a set, but  the collection of the worst math teachers in the world is \textbf{not} well-defined, and so is \textbf{not} a set.  In general, there are three ways to describe sets.  They are

\mnote{.55}{One thing that student evaluations teach us is that any given Mathematics instructor can be simultaneously the best and worst teacher ever, depending on who is completing the evaluation.}

\smallskip

\keyidea{idea:setdescription}{Ways to Describe Sets}{
\begin{enumerate}

\item \textbf{The Verbal Method:} Use a sentence to define a set.\index{set ! verbal description}

\item \textbf{The Roster Method:}  Begin with a left brace `$\{$', list each element of the set \textit{only once} and then end with a right brace `$\}$'.\index{set ! roster method}

\item \textbf{The Set-Builder Method:} A combination of the verbal and roster methods using a ``dummy variable'' such as $x$.\index{set ! set-builder notation}\index{set-builder notation}

\end{enumerate}
}
\smallskip

For example, let $S$ be the set described \textit{verbally} as the set of letters that make up the word ``pronghorns''.  A  \sword{roster} description of $S$ would be  $\left\{ p, r, o, n, g, h, s \right\}$. Note that we listed `r', `o', and `n' only once, even though they appear twice in ``pronghorns.''  Also, the \textit{order} of the elements doesn't matter, so $\left\{ o, n, p, r, g, s, h \right\}$ is also a roster description of $S$.   A \sword{set-builder} description of $S$ is: \[ \{ x \, | \, \mbox{$x$ is a
letter in the word ``pronghorns''.}\} \]

The way to read this is: `The set of elements $x$ \underline{such that} $x$ is a letter in the word ``pronghorns.'''   In each of the above cases, we may use the familiar equals sign `$=$' and write  $S = \left\{  p, r, o, n, g, h, s  \right\}$ or $S = \{ x \, | \, \mbox{$x$ is a
letter in the word ``pronghorns''.}\}$.  Clearly $r$ is in $S$ and $q$ is not in $S$.  We express these sentiments mathematically by writing  $r \in S$ and $q \notin S$. 

More precisely, we have the following.

\medskip

\definition{notationforsetinclusion}{Notation for set inclusion}{ Let $A$ be a set.

\begin{itemize}

\item If $x$ is an element of $A$ then we write $x \in A$\index{set ! inclusion}\index{$\in$} which is read `$x$ is in $A$'.

\item If $x$ is \emph{not} an element of $A$ then we write $x \notin A$\index{set ! exclusion}\index{$\notin$} which is read `$x$ is not in $A$'.

\end{itemize}
}

\medskip

Now let's consider the set $C =  \{ x \, | \, \mbox{$x$ is a consonant in the word ``pronghorns''}\}$.  A roster description of $C$ is  $C = \{ p, r, n, g, h, s\}$.  Note that by construction, every element of $C$ is also in $S$.  We express this relationship by stating that the set $C$ is a \textbf{subset} of the set $S$, which is written in symbols as $C \subseteq S$.  The more formal definition is given below.

\medskip

\definition{subsetdef}{Subset}{
Given sets $A$ and $B$, we say that the set $A$ is a \textbf{subset}\index{subset ! definition of} of the set $B$ and write `$A \subseteq B$' if every element in $A$ is also an element of $B$.  
}

\medskip

Note that in our example above $C \subseteq S$, but not vice-versa, since $o \in S$ but $o \notin C$.  Additionally, the set of vowels $V = \{ a, e, i, o, u\}$, while it does have an element in common with $S$, is not a subset of $S$. (As an added note,  $S$ is not a subset of $V$, either.)  We could, however, \textit{build} a set which contains both $S$ and $V$ as subsets by gathering all of the elements in both $S$ and $V$ together into a single set, say $U = \{ p, r, o, n, g, h, s, a, e, i, u\}$.   Then $S \subseteq U$ and $V \subseteq U$.  The set $U$ we have built is called the \textbf{union} of the sets $S$ and $V$ and is denoted $S \cup V$.    Furthermore, $S$ and $V$ aren't completely \textit{different}\footnote{Since the word `different' could be ambiguous, mathematicians use the word \textit{disjoint} to refer to two sets that have no elements in common.} sets since they both contain the letter `o.'  The \textbf{intersection} of two sets is the set of elements (if any) the two sets have in common. In this case, the intersection of $S$ and $V$ is $\{ o\}$, written $S \cap V = \{ o \}$.  We formalize these ideas below.

\medskip

\definition{intersectionunion}{Intersection and Union}{
 Suppose $A$ and $B$ are sets.

\begin{itemize}

\item The \textbf{intersection}\index{set ! intersection}\index{intersection of two sets} of $A$ and $B$ is $A \cap B = \{ x \, | \, x \in A \, \text{and} \,\, x \in B \}$

\item The \textbf{union}\index{set ! union}\index{union of two sets} of $A$ and $B$ is $A \cup B = \{ x \, | \, x \in A \, \text{or} \,\, x \in B \, \, \text{(or both)} \}$

\end{itemize}
}

\medskip

The key words in Definition \ref{intersectionunion} to focus on are the conjunctions:  `intersection' corresponds to `and' meaning the elements have to be in \textit{both} sets to be in the intersection, whereas `union' corresponds to `or' meaning the elements have to be in one set, or the other set (or both).  In other words, to belong to the union of two sets an element must belong to \textit{at least one} of them.

\smallskip

Returning to the sets $C$ and $V$  above, $C \cup V = \{ p, r, n, g, h, s, a, e, i, o, u\}$. When it comes to their intersection, however, we run into a bit of notational awkwardness since $C$ and $V$ have no elements in common.  While we could write $C \cap V = \{ \}$, this sort of thing happens often enough that we give the set with no elements a name. 

\medskip

\definition{emptysetdefn}{Empty set}{
The \textbf{Empty Set} $\emptyset$ is the set which contains no elements.\index{set ! empty}\index{empty set}  That is, \[\emptyset=\{ \}=\{x\,|\,\mbox{$x \neq x$}\}.\]  
}

\medskip

As promised, the empty set is the set containing no elements since no matter what `$x$' is, `$x = x$.'  Like the number `$0$,'  the empty set plays a vital role in mathematics. We introduce it here more as a symbol of convenience as opposed to a contrivance.
\mnote{.85}{The full extent of the empty set's role will not be explored in this text, but it is of fundamental importance in Set Theory. In fact, the empty set can be used to generate numbers -  mathematicians can create something from nothing! If you're interested, read about the \href{https://en.wikipedia.org/wiki/Natural_number\#von_Neumann_construction}{von Neumann construction of the natural numbers} or consider signing up for Math 2000.}   
Using this new bit of notation, we have for the sets $C$ and $V$ above that $C \cap V = \emptyset$.    A nice way to visualize relationships between sets and set operations is to draw a  \href{http://en.wikipedia.org/wiki/Venn_diagram}{\underline{\textbf{Venn Diagram}}}\index{diagram ! Venn Diagram}\index{Venn Diagram}.  A Venn Diagram for the sets $S$, $C$ and $V$ is drawn in Figure \ref{venndiagram}.

\mtable{.68}{A Venn diagram for $C$, $S$, and $V$}{venndiagram}{
\begin{tabular}{c}
\myincludegraphics[scale=0.8]{figures/RelationsandFunctionsGraphics/SetTheory-1}
\end{tabular}
}

In Figure \ref{venndiagram} we have three circles - one for each of the sets $C$, $S$ and $V$.  We visualize the area enclosed by each of these circles as the elements of each set.  Here, we've spelled out the elements for definitiveness.  Notice that the circle representing the set $C$ is completely inside the circle representing $S$.  This is a geometric way of  showing that $C \subseteq S$.  Also, notice that the circles representing $S$ and $V$ overlap on the letter `o'.  This common region is how we visualize $S \cap V$.  Notice that since $C \cap V = \emptyset$, the circles which represent $C$ and $V$ have no overlap whatsoever.  

\medskip

All of these circles lie in a rectangle labelled $U$ (for `universal' set).  A universal set contains all of the elements under discussion, so it could always be taken as the union of all of the sets in question, or an even larger set.  In this case, we could take $U = S \cup V$ or $U$ as the set of letters in the entire alphabet.    The usual triptych of Venn Diagrams indicating generic sets $A$ and  $B$ along with $A \cap B$ and $A \cup B$ is given below.

(The reader may well wonder if there is an ultimate universal set which contains \textit{everything}.  The short answer is `no'. Our definition of a set turns out to be overly simplistic, but correcting this takes us well beyond the confines of this course. If you want the longer answer, you can begin by reading about \href{http://en.wikipedia.org/wiki/Russell's_paradox}{\underline{Russell's Paradox}} on Wikipedia.)

\mtable{.32}{Venn diagrams for intersection and union}{fig:intunion}{
\begin{tabular}{c}
\myincludegraphics[scale=0.8]{figures/RelationsandFunctionsGraphics/SetTheory-2}\\
Sets $A$ and $B$.\\
\\
\myincludegraphics[scale=0.8]{figures/RelationsandFunctionsGraphics/SetTheory-3}\\
$A\cap B$ is shaded.\\
\\
\myincludegraphics[scale=0.8]{figures/RelationsandFunctionsGraphics/SetTheory-4}\\
$A\cup B$ is shaded.
\end{tabular}
}

\subsection{Sets of Real Numbers}
\label{SetsofRealNumbers}

The playground for most of this text is the set of \textbf{Real Numbers}.  Many quantities in the `real world' can be quantified using real numbers: the temperature at a given time, the revenue generated by selling a certain number of products and the maximum population of Sasquatch which can inhabit a particular region are just three basic examples.   A succinct, but nonetheless incomplete definition of a real number is given below.

\medskip

\definition{realnumberdefn}{The real numbers}{

A \sword{real number}\index{real number ! set of}\index{real number ! definition of} is any number which possesses a decimal representation.  The set of real numbers is denoted by the character $\mathbb{R}$.
}

\medskip

Certain subsets of the real numbers are worthy of note and are listed below.  In more advanced courses like Analysis, you learn that the real numbers can be {\em constructed} from the rational numbers, which in turn can be constructed from the integers (which themselves come from the natural numbers, which in turn can be defined as sets...).

\medskip

\phantomsection
\label{setsofnumbersboxonthispage}
\setboxwidth{30pt}
\noindent\hskip-12pt
\begin{minipage}{1.2\linewidth}
\definition{def:numbersets}{Sets of Numbers}{\index{set ! sets of numbers}

\begin{enumerate}

\item The \sword{Empty Set}:\index{set ! empty}\index{empty set} $\emptyset=\{ \}=\{x\,|\,\mbox{$x
\neq x$}\}$.  This is the set with no elements.  Like the number `$0$,' it  plays a vital role in mathematics.

\item The \sword{Natural Numbers}:\index{natural number ! set of}\index{natural number ! definition of} $\mathbb N= \{ 1, 2, 3,  \ldots\}$ The periods of ellipsis here indicate that the natural numbers contain $1$, $2$, $3$, `and so forth'.

%\item The \sword{Whole Numbers}:\index{whole number ! set of}\index{whole number ! definition of} $\mathbb W = \{ 0, 1, 2, \ldots \}$

\item The \sword{Integers}:\index{integer ! set of}\index{integer ! definition of} $\mathbb Z=\{ \ldots, -3, -2, -1, 0, 1, 2, 3, \ldots \}$

\item The \sword{Rational Numbers}:\index{rational number ! set of}\index{rational number ! definition of} $\mathbb Q=\left\{\frac{a}{b} \, | \, a \in \mathbb Z \, \mbox{and} \, b \in \mathbb Z \right\}$.  \underline{Ratio}nal numbers are the \underline{ratio}s of integers (provided the denominator is not zero!)  It turns out that another way to describe the rational numbers is: \[\mathbb Q=\{x\,|\,\mbox{$x$ possesses a repeating or terminating decimal representation.}\}\]

\item The \sword{Real Numbers}:\index{real number ! set of}\index{real number ! definition of} $\mathbb R = \{ x\,|\,\mbox{$x$ possesses a decimal representation.}\}$

\item The \sword{Irrational Numbers}:\index{irrational number ! set of}\index{irrational number ! definition of} $\mathbb P = \{x\,|\,\mbox{$x$ is a non-rational real number.}\}$  Said another way, an \underline{ir}rational number is a decimal which neither repeats nor terminates.

\item The \sword{Complex Numbers}:\index{complex number ! set of}\index{complex number ! definition of} $\mathbb C=\{a+bi\,|\,\mbox{$a$,$b \in \mathbb R$ and $i=\sqrt{-1}$}\}$  (We will not deal with complex numbers in Math 1010, although they usually make an appearance in Math 1410.)

\end{enumerate}
}
\end{minipage}
\restoreboxwidth
%\enlargethispage{.05in}

\medskip

\mnote{.65}{An example of a number with a repeating decimal expansion is $a=2.13234234234\ldots$. This is rational since $100a = 213.2342342342...$, and $100000a = 213234.234234...$ so $99900a = 100000a-100a = 213021$. This gives us the rational expression $a = \dfrac{213021}{99900}$.}
\mnote{.52}{The classic example of an irrational number is the number $\pi$ (See Section \ref{Angles}), but numbers like $\sqrt{2}$ and $0.101001000100001\ldots$ are other fine representatives.}

\medskip


It is important to note that every natural number is a whole number is an integer.   Each integer is a rational number (take $b =1$ in the above definition for $\mathbb Q$) and the rational numbers are all real numbers, since they possess decimal representations (via long division!). If we take $b=0$ in the above definition of $\mathbb C$, we see that every real number is a complex number.  In this sense, the sets $\mathbb N$, $\mathbb Z$, $\mathbb Q$, $\mathbb R$, and $\mathbb C$ are `nested' like \href{http://en.wikipedia.org/wiki/Matryoshka_doll}{\underline{Matryoshka dolls}}. More formally, these sets form a subset chain:  $\mathbb{N} \subseteq \mathbb Z \subseteq \mathbb{Q} \subseteq \mathbb{R}$.  The reader is encouraged to sketch a Venn Diagram depicting $\mathbb{R}$ and all of the subsets mentioned above.  It is time for an example.

\medskip


\example{numbersetex}{Sets of real numbers}{
\begin{enumerate}

\item  Write a roster description for $P = \{ 2^{n} \, | \, n \in \mathbb{N} \}$  and $E = \{ 2n \, | \, n \in \mathbb Z \}$.

\item Write a verbal description for $S = \{ x^2 \, | \, x \in \mathbb{R} \}$.

\item Let $A = \{-117, \frac{4}{5}, 0.20\overline{2002}, 0.202002000200002 \ldots\}$. 

Which elements of $A$ are natural numbers?  Rational numbers?  Real numbers?


%\item  What is another name for $\mathbb{N} \cup \mathbb{Q}$?  What about  $\mathbb{Q} \cup \mathbb P$?

\end{enumerate}
}
{
\begin{enumerate}

\item  To find a roster description for these sets, we need to list their elements.   Starting with $P = \{ 2^{n} \, | \, n \in \mathbb{N} \}$, we substitute natural number values $n$ into the formula $2^n$.  For $n = 1$ we get $2^1 = 2$,  for $n = 2$ we get $2^2 = 4$, for $n = 3$ we get $2^3 = 8$ and for $n = 4$ we get $2^4 = 16$.  Hence  $P$ describes the powers of $2$, so a roster description for $P$ is $P = \{ 2, 4, 8, 16, \ldots \}$ where the `$\ldots$' indicates the that pattern continues.

\mnote{.8}{This isn't the most \textit{precise} way to describe this set - it's always dangerous to use `$\ldots$' since we assume that the pattern is clearly demonstrated and thus made evident to the reader.  Formulas are more precise because the pattern is clear.}  

\smallskip

Proceeding in the same way, we generate elements in $E = \{ 2n \, | \, n \in \mathbb Z \}$ by plugging in integer values of $n$ into the formula $2n$.  Starting with $n = 0$ we obtain $2(0) = 0$.  For $n = 1$ we get $2(1) = 2$, for $n = -1$ we get $2(-1) = -2$ for $n = 2$, we get $2(2) = 4$ and for $n = -2$ we get $2(-2) = -4$.  As $n$  moves through the integers, $2n$ produces all of the \textit{even} integers. A roster description for  $E$ is $E = \{ 0, \pm 2, \pm 4, \ldots \}$.

\mnote{.7}{It shouldn't be too surprising that $E$ is the set of all even integers, since an even integer is \textit{defined} to be an integer multiple of $2$.}

\item  One way to verbally describe $S$ is to say that $S$ is the `set of all squares of real numbers'.  While this isn't incorrect, we'd like to take this opportunity to delve a little deeper.  What makes the set $S = \{ x^2 \, | \, x \in \mathbb{R} \}$ a little trickier to wrangle than the sets $P$ or $E$ above is that the dummy variable here, $x$, runs through all \textit{real} numbers.  Unlike the natural numbers or the integers, the real numbers cannot be listed in any methodical way. Nevertheless, we can select some real numbers, square them and get a sense of what kind of numbers lie in $S$.  For $x = -2$, $x^2 = (-2)^2 = 4$ so $4$ is in $S$, as are $\left(\frac{3}{2}\right)^2 = \frac{9}{4}$ and $(\sqrt{117})^2 = 117$.  Even things like $(-\pi)^2$ and $(0.101001000100001 \ldots)^2$ are in $S$.  

\mnote{.5}{The fact that the real numbers cannot be listed is a nontrivial statement.  Interested readers are directed to a discussion of \href{http://en.wikipedia.org/wiki/Cantor's_diagonal_argument}{\underline{Cantor's Diagonal Argument}}.} 

\smallskip

So suppose $s \in S$.  What can be said about $s$?  We know there is some real number $x$ so that $s = x^2$.  Since $x^2 \geq 0$ for any real number $x$, we know $s \geq 0$.  This tells us that everything in $S$ is a non-negative real number.  This begs the question:  are \underline{all} of the non-negative real numbers in $S$?  Suppose $n$ is a non-negative real number, that is, $n \geq 0$.  If $n$ were in $S$, there would be a real number $x$ so that $x^2=n$.  As you may recall, we can solve $x^2 = n$ by `extracting square roots':  $x = \pm \sqrt{n}$.  Since $n \geq 0$, $\sqrt{n}$ is a real number. Moreover, $(\sqrt{n})^2 = n$ so $n$ is the square of a real number which means $n \in S$. Hence, $S$ is the set of non-negative real numbers.

\item  The set $A$ contains no natural numbers. Clearly, $\frac{4}{5}$ is a rational number as is $-117$ (which can be written as $\frac{-117}{1}$). It's the last two numbers listed in $A$, $0.20\overline{2002}$ and $0.202002000200002 \ldots$, that warrant some discussion.  First, recall that the `line' over the digits $2002$ in $0.20\overline{2002}$ (called the vinculum) indicates that these digits repeat, so it is a rational number. As for the number $0.202002000200002 \ldots$, the `$\ldots$' indicates the pattern of adding an extra `$0$' followed by a `$2$' is what defines this real number.  Despite the fact there is a \textit{pattern} to this decimal, this decimal is \textit{not repeating}, so it is not a rational number - it is, in fact, an irrational number.  All of the elements of $A$ are real numbers, since all of them can be expressed as decimals (remember that $\frac{4}{5} = 0.8$).\qed

\item  The set $\mathbb{N} \cup \mathbb{Q} = \{ x \, | \, \text{$x \in \mathbb{N}$ or $x \in \mathbb{Q}$} \}$ is the union of the set of natural numbers with the set of rational numbers.  Since every natural number is a rational number, $\mathbb{N}$ doesn't contribute any new elements to $\mathbb{Q}$, so $\mathbb{N} \cup \mathbb{Q} = \mathbb{Q}$. For the  set $\mathbb{Q} \cup \mathbb P$, we note that every real number is either rational or not, hence $\mathbb{Q} \cup \mathbb P = \mathbb{R}$, pretty much by the definition of the set $\mathbb P$.  


\end{enumerate}
}
\pagebreak

As you may recall, we often visualize the set of real numbers $\mathbb{R}$ as a line where each point on the line corresponds to one and only one real number.  Given two different real numbers $a$ and $b$,  we write $a < b$ if $a$ is located to the left of $b$ on the number line, as shown in Figure \ref{fig:numline}.




While this notion seems innocuous, it is worth pointing out that this convention is rooted in two deep properties of real numbers.  The first property is that $\mathbb{R}$ is  \href{http://en.wikipedia.org/wiki/Complete_metric_space}{\underline{complete}}. This means that there are no `holes' or `gaps' in the real number line.\footnote{Alas, this intuitive feel for what it means to be `complete' is as good as it gets at this level.  Completeness does get a much more precise meaning later in courses like Analysis and Topology.} Another way to think about this is that if you choose any two distinct (different) real numbers, and look between them, you'll find a solid line segment (or interval) consisting of infinitely many real numbers.  

\mfigure{.75}{The real number line with two numbers $a$ and $b$, where $a<b$.}{fig:numline}{figures/RelationsandFunctionsGraphics/SetTheory-5}

The next result tells us what types of numbers we can expect to find.

\medskip

\theorem{densityofqandp}{Density Property of $\mathbb{Q}$ and $\mathbb P$ in $\mathbb{R}$}{

Between any two distinct real numbers, there is at least one rational number and irrational number.  It then follows that between any two distinct real numbers there will be \underline{infinitely many} rational and irrational numbers.
}

\medskip

The root word `dense' here communicates the idea that rationals and irrationals are `thoroughly mixed' into $\mathbb{R}$.   The reader is encouraged to think about how one would find both a rational and an irrational number between, say, $0.9999$ and $1$. Once you've done that, ask yourself whether there is any difference between the numbers $0.\overline{9}$ and $1$. 

\smallskip

The second property $\mathbb{R}$ possesses that lets us view it as a line is that the set is \href{http://en.wikipedia.org/wiki/Total_order}{\underline{totally ordered}}. This means that given any two real numbers $a$ and $b$, either $a < b$, $a > b$ or $a = b$ which allows us to arrange the numbers from least (left) to greatest (right). You may have heard this property given as the `Law of Trichotomy'.\index{trichotomy}

\medskip

\definition{trichotomy}{Law of Trichotomy}{
If $a$ and $b$ are real numbers then {\bf exactly one} of the following statements is true: \vspace{-.15in} \[ \begin{array}{lclcl} a < b & \hspace*{1.25in} & a > b & \hspace*{1.25in} & a = b \end{array} \]
}
\mnote{.5}{The Law of Trichotomy, strictly speaking, is an \textit{axiom} of the real numbers: a basic requirement that we assume to be true. However, in any \textit{construction} of the real, such as the method of \href{https://en.wikipedia.org/wiki/Dedekind_cut}{Dedekind cuts}, it is necessary to \textit{prove} that the Law of Trichotomy is satisfied.}

\medskip

 Segments of the real number line are called \sword{intervals}\index{interval ! definition of} of numbers. Below is a summary of the so-called \sword{interval notation}\index{interval ! notation for} associated with given sets of numbers.  For intervals with finite endpoints, we list the left endpoint, then the right endpoint.  We use square brackets, `$[$' or `$]$', if the endpoint is included in the interval and use a filled-in or `closed' dot to indicate membership in the interval. Otherwise, we use parentheses, `$($' or `$)$' and an `open' circle to indicate that the endpoint is not part of the set.  If the interval does not have finite endpoints, we use the symbols $-\infty$ to indicate that the interval extends indefinitely to the left and $\infty$ to indicate that the interval extends indefinitely to the right.  Since infinity is a concept, and not a number, we always use parentheses when using these symbols in interval notation, and use an appropriate arrow to indicate that the interval extends indefinitely in one (or both) directions.

\medskip
\setboxwidth{10pt}
\noindent\begin{minipage}{1.1\linewidth}
\definition{def:intervalnot}{Interval Notation}{
Let $a$ and $b$ be real numbers with $a<b$.

\smallskip

%\begin{center}

\noindent\hskip-10pt\begin{tabular}{|c|c|c|} \hline

Set of Real Numbers & Interval Notation &  Region on the Real Number Line  \\
\hline

 &  & \\
\shortstack{$\{x\,|\,a<x<b\}$ \\ \hfill}& \shortstack{$(a,b)$ \\ \hfill} & 

\myincludegraphics{figures/RelationsandFunctionsGraphics/CartesianPlane-1}  \\ \hline

& &  \\
\shortstack{$\{x\,|\,a\leq x<b\}$ \\ \hfill}& \shortstack{$[a,b)$ \\ \hfill} & 

\myincludegraphics{figures/RelationsandFunctionsGraphics/CartesianPlane-2}   \\
\hline

 &  & \\
\shortstack{$\{x\,|\,a<x\leq b\}$ \\ \hfill}&\shortstack{$(a,b]$ \\ \hfill} & 

\myincludegraphics{figures/RelationsandFunctionsGraphics/CartesianPlane-3}   \\
\hline

 &  & \\
\shortstack{$\{x\,|\,a\leq x \leq b\}$ \\ \hfill}& \shortstack{$[a,b]$ \\ \hfill}& 

\myincludegraphics{figures/RelationsandFunctionsGraphics/CartesianPlane-4}   \\
\hline

 & & \\
\shortstack{$\{x\,| \, x<b\}$ \\ \hfill}& \shortstack{$(-\infty,b)$ \\ \hfill}& 

\myincludegraphics{figures/RelationsandFunctionsGraphics/CartesianPlane-5}   \\
\hline


&  & \\

\shortstack{$\{x\,| \, x \leq b\}$ \\ \hfill} & \shortstack{$(-\infty,b]$ \\ \hfill}& 

\myincludegraphics{figures/RelationsandFunctionsGraphics/CartesianPlane-6}   \\
\hline

 &  & \\
\shortstack{$\{x\,| \, x>a\}$ \\ \hfill}& \shortstack{$(a,\infty)$ \\ \hfill}& 

\myincludegraphics{figures/RelationsandFunctionsGraphics/CartesianPlane-7}   \\
\hline

 &  & \\
\shortstack{$\{x\,| \, x \geq a \}$ \\ \hfill}& \shortstack{$[a,\infty)$ \\ \hfill} & 

\myincludegraphics{figures/RelationsandFunctionsGraphics/CartesianPlane-8}   \\
\hline

&  & \\
\shortstack{$\mathbb R$ \\ \hfill}& \shortstack{$(-\infty,\infty)$ \\ \hfill} & 

\myincludegraphics{figures/RelationsandFunctionsGraphics/CartesianPlane-9}   \\
\hline

\end{tabular}

%\end{center}
}
\end{minipage}
%\pagebreak
\restoreboxwidth

\medskip


As you can glean from the table, for intervals with finite endpoints we start by writing `left endpoint, right endpoint'.  We use square brackets, `$[$' or `$]$', if the endpoint is included in the interval. This corresponds to a `filled-in' or `closed' dot on the number line to indicate that the number is included in the set.  Otherwise, we use parentheses, `$($' or `$)$' that correspond to an `open' circle which indicates that the endpoint is not part of the set.  If the interval does not have finite endpoints, we use the symbol $-\infty$ to indicate that the interval extends indefinitely to the left and the symbol $\infty$ to indicate that the interval extends indefinitely to the right.  Since infinity is a concept, and not a number, we always use parentheses when using these symbols in interval notation, and use the appropriate arrow to indicate that the interval extends indefinitely in one or both directions. 

\mnote{.5}{The importance of understanding interval notation in Calculus cannot be overstated. If you don't find yourself getting the hang of it through repeated use, you may need to take the time to just memorize this chart.} 

Let's do a few examples to make sure we have the hang of the notation:

\medskip




\begin{center}
\begin{tabular}{|c|c|c|} \hline

Set of Real Numbers & Interval Notation &  Region on the Real Number Line  \\
\hline

& &  \\
\shortstack{$\{x\,|\,1\leq x< 3\}$ \\ \hfill} & \shortstack{$[1,3)$ \\ \hfill} & 

\myincludegraphics{figures/RelationsandFunctionsGraphics/CartesianPlane-10}  \\
\hline

 &  & \\
\shortstack{$\{x\,|\,-1\leq x \leq 4\}$ \\ \hfill}& \shortstack{$[-1,4]$ \\ \hfill} & 

\myincludegraphics{figures/RelationsandFunctionsGraphics/CartesianPlane-11}   \\
\hline

&  & \\

\shortstack{$\{x\,| \, x \leq 5 \}$ \\ \hfill} & \shortstack{$(-\infty, 5]$ \\ \hfill} &

\myincludegraphics{figures/RelationsandFunctionsGraphics/CartesianPlane-12}   \\
\hline

 &  & \\
\shortstack{$\{x\,| \, x > -2 \}$ \\ \hfill} & \shortstack{$(-2, \infty)$ \\ \hfill} &  

\myincludegraphics{figures/RelationsandFunctionsGraphics/CartesianPlane-13}   \\
\hline

\end{tabular}

\end{center}


\medskip

We defined the intersection and union of arbitrary sets in Definition \ref{intersectionunion}. Recall that  the union of two sets consists of the totality of the elements in each of the sets, collected together. For example,  if $A = \{ 1,2,3 \}$ and $B = \{2,4,6 \}$, then $A \cap B = \{2\}$ and $A \cup B = \{1,2,3,4,6\}$.   If $A = [-5,3)$ and $B = (1, \infty)$, then we can find $A \cap B$ and $A\cup B$ graphically.  To find $A\cap B$, we shade  the overlap of the two and obtain $A \cap B = (1,3)$.  To find $A \cup B$, we shade each of $A$ and $B$ and describe the resulting shaded region to find  $A \cup B = [-5,\infty)$.

\mtable{.75}{Union and intersection of intervals}{fig:venn1}{%
\begin{tabular}{c}
\myincludegraphics{figures/RelationsandFunctionsGraphics/CartesianPlane-14}\\
$A = [-5,3)$,  $B = (1, \infty)$\\
\\
\myincludegraphics{figures/RelationsandFunctionsGraphics/CartesianPlane-15}\\
$A \cap B = (1,3)$\\
\\
\myincludegraphics{figures/RelationsandFunctionsGraphics/CartesianPlane-16}\\
$A \cup B = [-5,\infty)$
\end{tabular}
}


While both intersection and union are important, we have more occasion to use union in this text than intersection, simply because most of the sets of real numbers we will be working with are either intervals or are unions of intervals, as the following example illustrates.

\medskip


\example{unionex}{Expressing sets as unions of intervals}{
Express the following sets of numbers using interval notation.

\begin{multicols}{2}

\begin{enumerate}

\item  $\{ x \, | \, x \leq -2 \, \, \text{or} \, \,  x \geq 2 \}$

\item  $\{ x \, | \, x \neq 3 \}$

%\setcounter{HW}{\value{enumi}}

\end{enumerate}

\end{multicols}

\begin{multicols}{2}

\begin{enumerate}

\setcounter{enumi}{2}

\item  $\{ x \, | \, x \neq \pm 3 \}$

\item  $\{ x \, | \, -1 < x \leq 3 \,\, \text{or} \,\, x = 5\}$

\end{enumerate}

\end{multicols}
}
{
\begin{enumerate}

\item  The best way to proceed here is to graph the set of numbers on the number line and glean the answer from it.  The inequality $x \leq -2$ corresponds to the interval $(-\infty, -2]$ and the inequality $x \geq 2$ corresponds to the interval $[2, \infty)$.  Since we are looking to describe the real numbers $x$ in one of these \textit{or} the other, we have $\{ x \, | \, x \leq -2 \, \, \text{or} \, \,  x \geq 2 \} = (-\infty, -2] \cup [2, \infty)$.

\mfigure{.54}{The set $(-\infty, -2] \cup [2, \infty)$}{fig:eg1-1}{figures/RelationsandFunctionsGraphics/CartesianPlane-17}

\item For the set $\{ x \, | \, x \neq 3 \}$, we shade the entire real number line except $x=3$, where we leave an open circle.  This divides the real number line into two intervals, $(-\infty, 3)$ and $(3,\infty)$.  Since the values of $x$ could be in either one of these intervals \textit{or} the other, we have that $\{ x \, | \, x \neq 3 \} = (-\infty, 3) \cup (3,\infty)$
 
\mfigure{.46}{The set  $(-\infty, 3) \cup (3, \infty)$}{fig:eg1-2}{figures/RelationsandFunctionsGraphics/CartesianPlane-18}

\item  For the set $\{ x \, | \, x \neq \pm 3 \}$, we proceed as before and exclude both $x=3$ and $x=-3$ from our set.  This breaks the number line into \textit{three} intervals, $(-\infty, -3)$, $(-3,3)$ and $(3, \infty)$.   Since the set describes real numbers which come from the first, second \textit{or} third interval, we have $\{ x \, | \, x \neq \pm 3 \} = (-\infty, -3) \cup (-3,3) \cup (3, \infty)$.

\mfigure{.38}{The set $(-\infty, -3) \cup (-3,3) \cup (3, \infty)$}{fig:eg1-3}{figures/RelationsandFunctionsGraphics/CartesianPlane-19}


\item  Graphing the set $\{ x \, | \, -1 < x \leq 3 \,\, \text{or} \,\, x = 5\}$, we get one interval, $(-1,3]$ along with a single number, or point, $\{ 5\}$.  While we \textit{could} express the latter as $[5,5]$ (Can you see why?), we choose to write our answer as $\{ x \, | \, -1 < x \leq 3 \,\, \text{or} \,\, x = 5\} = (-1,3] \cup \{ 5\}$.

\mfigure{.3}{The set $(-1,3] \cup \{ 5\}$}{fig:eg1-4}{figures/RelationsandFunctionsGraphics/CartesianPlane-20}
\end{enumerate}
}


\printexercises{exercises_pre/00_01_exercises}
