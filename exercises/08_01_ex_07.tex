{ Consider the pendulum below.  Ignoring air resistance, the angular displacement of the pendulum from the vertical position, $\theta$, can be modelled as a sinusoid.\footnote{Provided $\theta$ is kept `small.'  Carl remembers the `Rule of Thumb' as being $20^{\circ}$ or less.  Check with your friendly neighborhood physicist to make sure.}

\begin{center}
\myincludegraphics{figures/AppExtGraphics/Sinusoid-9}
\end{center} 

The amplitude of the sinusoid is the same as the initial angular displacement, $\theta_{\text{\tiny $0$}}$, of the pendulum and the  period of the motion is given by \[T = 2\pi \sqrt{\dfrac{l}{g}}\] where $l$ is the length of the pendulum and $g$ is the acceleration due to gravity. 

\begin{enumerate}
\item   Find a sinusoid which gives the angular displacement $\theta$ as a function of time, $t$. Arrange things so $\theta(0) = \theta_{\text{\tiny $0$}}$. 
\item  In Exercise \ref{pendulumproblem} section \ref{AlgebraicFunctions}, you found the length of the pendulum needed in Jeff's antique Seth-Thomas clock to ensure the period of the pendulum is $\frac{1}{2}$ of a second. Assuming the initial displacement of the pendulum is $15^{\circ}$, find a sinusoid which models the displacement of the pendulum $\theta$ as a function of time, $t$, in seconds. 
\end{enumerate}}
{\begin{enumerate}
 \item  $\theta(t) = \theta_{\text{\tiny $0$}} \sin\left(\sqrt{\frac{g}{l}}\, t + \frac{\pi}{2}\right)$ 
 \item  $\theta(t) = \frac{\pi}{12} \sin\left(4\pi t + \frac{\pi}{2}\right)$ 
 \end{enumerate}}
