{\phantomsection
\label{bearings}
\index{bearings} 
(Another Classic Application: Bearings)  In Exercises}
{,  we introduce and work with the navigation tool known as bearings.  Simply put, a bearing is the direction you are heading according to a compass.  The classic nomenclature for bearings, however, is not given as an angle in standard position, so we must first understand the notation.  A bearing is given as an acute angle of rotation (to the east or to the west) away from the north-south (up and down) line of a compass rose.  For example, N$40^{\circ}$E (read ``$40^{\circ}$ east of north'') is a bearing which is rotated clockwise $40^{\circ}$ from due north.  If we imagine standing at the origin in the Cartesian Plane, this bearing would have us heading into Quadrant I along the terminal side of $\theta = 50^{\circ}$.  Similarly, S$50^{\circ}$W would point into Quadrant III along the terminal side of $\theta = 220^{\circ}$ because we started out pointing due south (along $\theta = 270^{\circ}$) and rotated clockwise $50^{\circ}$ back to $220^{\circ}$.  Counter-clockwise rotations would be found in the bearings N$60^{\circ}$W (which is on the terminal side of $\theta = 150^{\circ}$) and S$27^{\circ}$E (which lies along the terminal side of $\theta = 297^{\circ}$).  These four bearings are drawn in the plane below.
\begin{center}
 \myincludegraphics{figures/AppExtGraphics/LawofSines-21}
\end{center}
The cardinal directions north, south, east and west are usually not given as bearings in the fashion described above, but rather, one just refers to them as `due north', `due south', `due east' and `due west', respectively, and it is assumed that you know which quadrantal angle goes with each cardinal direction.  (Hint: Look at the diagram above.)  }
\exinput{exercises/08_02_ex_25}
\exinput{exercises/08_02_ex_26}
\exinput{exercises/08_02_ex_27}
\exinput{exercises/08_02_ex_28}
\exinput{exercises/08_02_ex_29}
\exinput{exercises/08_02_ex_30}
\exinput{exercises/08_02_ex_31}