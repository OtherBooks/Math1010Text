{$\displaystyle \lim_{x\to 4} x^2+x-5 = 15$}
{Let $\epsilon >0$ be given. We wish to find $\delta >0$ such that when $|x-4|<\delta$, $|f(x)-15|<\epsilon$. 

Consider $|f(x)-15|<\epsilon$, keeping in  mind we want to make a statement about $|x-4|$:
\begin{gather*}
|f(x) -15 | < \epsilon \\
|x^2+x-5 -15 |<\epsilon \\
| x^2+x-20 | < \epsilon \\
| x-4 |\cdot|x+5| < \epsilon \\
| x-4 | < \epsilon/|x+5| \\
\end{gather*}
Since $x$ is near 4, we can safely assume that, for instance, $3<x<5$. Thus
\begin{gather*}
3+5<x+5<5+5 \\
8 < x+5 < 10 \\
\frac{1}{10} < \frac{1}{x+5} < \frac{1}{8} \\
\frac{\epsilon}{10} < \frac{\epsilon}{x+5} < \frac{\epsilon}{8} \\
\end{gather*}
Let $\delta =\frac{\epsilon}{10}$. Then:
\begin{gather*}
|x-4|<\delta \\
|x-4| < \frac{\epsilon}{10}\\
|x-4| < \frac{\epsilon}{x+5}\\
|x-4|\cdot|x+5| < \frac{\epsilon}{x+5}\cdot|x+5|\\
\end{gather*}
Assuming $x$ is near 4, $x+5$ is positive and we can drop the absolute value signs on the right.
\begin{gather*}
|x-4|\cdot|x+5| < \frac{\epsilon}{x+5}\cdot(x+5)\\
|x^2+x-20| < \epsilon\\
|(x^2+x-5) -15| < \epsilon,
\end{gather*}
which is what we wanted to prove.
}


