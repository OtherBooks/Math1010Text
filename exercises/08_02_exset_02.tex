{\noindent \phantomsection
\label{gradeofroad}
(Another Classic Application:  Grade of a Road) The grade of a road is much like the pitch of a roof (See Example \ref{roofpitchex}) in that it expresses the ratio of rise/run.  In the case of a road, this ratio is always positive because it is measured going uphill and it is usually given as a percentage.  For example, a road which rises 7 feet for every 100 feet of (horizontal) forward progress is said to have a 7\% grade.  However, if we want to apply any Trigonometry to a story problem involving roads going uphill or downhill, we need to view the grade as an angle with respect to the horizontal.  In Exercises}
{, we first have you change road grades into angles and then use the Law of Sines in an application.}
\exinput{exercises/08_02_ex_22}
\exinput{exercises/08_02_ex_23}
\exinput{exercises/08_02_ex_24}