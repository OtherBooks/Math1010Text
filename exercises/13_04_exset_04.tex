{The \textit{moment of inertia} $I$ is a measure of the tendency of a lamina to resist rotating about an axis or continue to rotate about an axis. $I_x$ is the moment of inertia about the $x$-axis, $I_x$ is the moment of inertia about the $x$-axis, and $I_O$ is the moment of inertia about the origin. These are computed as follows:
\begin{itemize}
	\item $\ds I_x = \iint_R y^2\ dm$
	\item	$\ds I_y = \iint_R x^2\ dm$
	\item	$\ds I_O = \iint_R \big(x^2+y^2\big)\ dm$
\end{itemize} \noindent In Exercises} 
{,  a lamina corresponding to a planar region $R$ is given with a mass of 16 units. For each, compute $I_x$, $I_y$ and $I_O$.
}
\exinput{exercises/13_04_ex_27}
\exinput{exercises/13_04_ex_28}
\exinput{exercises/13_04_ex_29}
\exinput{exercises/13_04_ex_30}