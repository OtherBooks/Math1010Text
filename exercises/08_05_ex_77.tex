{ Use the Sum and Difference Identities in Theorem \ref{circularsumdifference} or the Half Angle Identities in Theorem \ref{halfangle} to express the three cube roots of $z=\sqrt{2} + i\sqrt{2}$ in rectangular form. (See Example \ref{nthrootscomplexex}, number \ref{halfanglecuberoot}.)}
{ Note: In the answers for $w_{\text{\tiny$0$}}$ and $w_{\text{\tiny$2$}}$ the first rectangular form comes from applying the appropriate Sum or Difference Identity ($\frac{\pi}{12} = \frac{\pi}{3} - \frac{\pi}{4}$ and $\frac{17\pi}{12} = \frac{2\pi}{3} + \frac{3\pi}{4}$, respectively) and the second comes from using the Half-Angle Identities. \\$w_{\text{\tiny$0$}} = \sqrt[3]{2} \operatorname{cis}\left(\frac{\pi}{12}\right) = \sqrt[3]{2}\left( \frac{\sqrt{6} + \sqrt{2}}{4} + i\left( \frac{\sqrt{6} - \sqrt{2}}{4} \right) \right) = \sqrt[3]{2}\left( \frac{\sqrt{2 + \sqrt{3}}}{2} + i\frac{\sqrt{2 - \sqrt{3}}}{2} \right)$ \\$w_{\text{\tiny$1$}} = \sqrt[3]{2} \operatorname{cis}\left(\frac{3\pi}{4}\right) = \sqrt[3]{2} \left( -\frac{\sqrt{2}}{2} + \frac{\sqrt{2}}{2}i \right)$ \\$w_{\text{\tiny$2$}} = \sqrt[3]{2} \operatorname{cis}\left(\frac{17\pi}{12}\right) = \sqrt[3]{2}\left( \frac{\sqrt{2} - \sqrt{6}}{4} + i\left( \frac{-\sqrt{2} - \sqrt{6}}{4} \right) \right) = \sqrt[3]{2}\left( \frac{\sqrt{2 - \sqrt{3}}}{2} + i\frac{\sqrt{2 + \sqrt{3}}}{2} \right)$}
