{Suppose there are two lakes located on a stream.  Clean
water flows into the first lake,
then the water from the first lake flows into the second lake, and then
water from the second lake flows further downstream.
The in and out flow from each lake is 500 litres per hour.
The first lake contains 100 thousand litres of water and the
second lake contains 200 thousand litres of water.
A truck with \unit[500]{kg} of toxic substance
crashes into the first lake.  Assume that the water is being continually
mixed perfectly by the stream.  a) Find the concentration of toxic substance
as a function of time in both lakes.  b) When will the
concentration in the first lake be below \unit[0.001]{kg} per litre?
c) When will the
concentration in the second lake be maximal?}
{(a) The first lake begins with 500 kg of the substance at time $t=0$, and no additional substance is added. If $x(t)$ denotes the amount of toxic substance at time $t$, we have $x(0)=500$. Since water flows out of the lake at a rate of 500 litres per hour, and there are 100,000 in the lake (this amount is assumed constant) we have $x'(t) = -\frac{500}{100000}x(t)$. Thus, $x(t) = 500e^{-t/200}$. Let $y(t)$ denote the amount of toxic substance in the second lake. We have $y(0)=0$. The substance is flowing in at a rate of $\frac{x(t)}{100000} kg/l\times 500 l/hour = \frac{x(t)}{200} = \frac{5}{2}e^{-t/200}$ kg/hour. The second lake has a volume of 200,000 litres, so the concentration of the substance in the second lake is $y(t)/200000$, and the water flows out at a rate of 500 litres per hour. This gives us the equation
\[
y'(t) = \frac{5}{2}e^{-t/200}-\frac{1}{400}y(t),
\]
which yields the solution $y(t) = 500e^{-t/400}-500e^{-3t/400}$.\\
(b) A concentration of 0.001 kg per litre corresponds to $x(t) = 100$, since there are 100,000 litres in the first lake. Setting $x(t)=100$ and solving for $t$ gives $t=200\ln 5 \approx 322$ hours.
(c) The maximum concentration in the second lake occurs after $t=200\ln 3 \approx 220$ hours.}