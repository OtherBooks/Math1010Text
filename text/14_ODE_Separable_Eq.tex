\section{Separable equations}\label{sec:separable}


When a differential equation is of the form
$y' = f(x)$,
we can just integrate:
$y = \int f(x) ~dx + C$. 
Unfortunately this method no longer works for the
general form of the equation
$y' = f(x,y)$.
Integrating both sides yields
\begin{equation*}
y = \int f(x,y) ~dx + C .
\end{equation*}
Notice the dependence on $y$ in the integral.

\subsection*{Separable equations}

Let us suppose that the equation is
\sword{separable}\index{differential equation!separable}\index{separable ODE}.
That is, let us consider
\begin{equation*}
y' = f(x)g(y) ,
\end{equation*}
for some functions $f(x)$ and $g(y)$.
Let us write the equation in the \sword{Leibniz notation}\index{Leibniz notation}
\begin{equation*}
\frac{dy}{dx} = f(x)g(y) .
\end{equation*}
Then we rewrite the equation as
\begin{equation*}
\frac{dy}{g(y)} = f(x) ~dx .
\end{equation*}
Now both sides look like something we can integrate.  We obtain
\begin{equation*}
\int \frac{dy}{g(y)} = \int f(x) ~dx + C .
\end{equation*}
If we can find closed form expressions
for these two integrals, we can, perhaps, solve for $y$.

\example{egh1}{A separable ODE}{
Solve the equation
\begin{equation*}
y' = xy .
\end{equation*}}
{
First note that $y=0$ is a solution,
so assume $y \not =0$ from now on.
Write the equation as $\frac{dy}{dx} = xy$, then
\begin{equation*}
\int \frac{dy}{y} = \int x~dx + C .
\end{equation*}
We compute the antiderivatives to get
\begin{equation*}
\ln \, \lvert y\rvert = \frac{x^2}{2} + C .
\end{equation*}
Or
\begin{equation*}
\lvert y \rvert = e^{\frac{x^2}{2} + C} = e^{\frac{x^2}{2}} e^C = D e^{\frac{x^2}{2}} ,
\end{equation*}
where $D > 0$ is some constant.  Because $y=0$ is a solution and because
of the absolute value we actually can write:
\begin{equation*}
y = D e^{\frac{x^2}{2}} ,
\end{equation*}
for any number $D$ (including zero or negative).

We check:
\begin{equation*}
y' = D x e^{\frac{x^2}{2}} = x \left( D e^{\frac{x^2}{2}} \right) = xy .
\end{equation*}
Yay!
}\\

We should be a little bit more careful with this method.  You may be worried 
that we were
integrating in two different variables.
We seemed to be
doing a different operation to each side.  Let us work this method out more
rigorously.  Take
\begin{equation*}
\frac{dy}{dx} = f(x)g(y) .
\end{equation*}
We rewrite the equation as follows.
Note that $y = y(x)$ is a function of $x$ and so is
$\frac{dy}{dx}$!
\begin{equation*}
\frac{1}{g(y)}~\frac{dy}{dx} = f(x) .
\end{equation*}
We integrate both sides with respect to $x$.
\begin{equation*}
\int \frac{1}{g(y)}~\frac{dy}{dx} ~dx = \int f(x) ~dx + C .
\end{equation*}
We can use the change of variables formula.
\begin{equation*}
\int \frac{1}{g(y)}~dy = \int f(x) ~dx + C .
\end{equation*}
And we are done.

\subsection*{Implicit solutions}

It is clear that we might sometimes get stuck even if we can do the
integration.  For example, take the separable equation
\begin{equation*}
y' = \frac{xy}{y^2+1} .
\end{equation*}
We separate variables,
\begin{equation*}
\frac{y^2+1}{y}~dy = \left(y+\frac{1}{y}\right)~dy = x~dx .
\end{equation*}
We integrate to get
\begin{equation*}
\frac{y^2}{2} + \ln \, \lvert y \rvert = \frac{x^2}{2} + C ,
\end{equation*}
or perhaps the easier looking expression (where $D = 2C$)
\begin{equation*}
y^2 + 2 \ln \, \lvert y\rvert = x^2 + D .
\end{equation*}
It is not easy to find the solution explicitly as it is hard to solve
for $y$.  We, therefore, leave the solution in this form and call
it an
\sword{implicit solution}\index{implicit solution}\index{solution!implicit}.
It is still
easy to check that an implicit solution satisfies the differential
equation.  In this case, we differentiate with respect to $x$
to get
\begin{equation*}
y'\left(2y + \frac{2}{y}\right) = 2x .
\end{equation*}
It is simple to see that the differential equation holds.
If you want to compute values
for $y$, you might have to be tricky.  For example, you can graph
$x$ as a function of $y$, and then flip your paper.  Computers are also good
at some of these tricks.

We note that the above equation also has the solution $y=0$.
The general solution is 
$y^2 + 2 \ln \, \lvert y \rvert = x^2 + C$ together with $y=0$.  These outlying solutions
such as $y=0$
are sometimes called \emph{singular solutions\index{singular solution}}.\\

\example{egh2}{An example with initial conditions}{
Solve $x^2y' = 1 - x^2+y^2 - x^2y^2$, $y(1) = 0$.}
{
First factor the right hand side to obtain
\begin{equation*}
x^2y' = (1 - x^2)(1+y^2) .
\end{equation*}
Separate variables, integrate, and solve for $y$.
\begin{align*}
\frac{y'}{1+y^2} & = \frac{1 - x^2}{x^2} , \\
\frac{y'}{1+y^2} & = \frac{1}{x^2} - 1 , \\
\operatorname{arctan} (y) & = \frac{-1}{x} - x + C , \\
y & = \tan \left(\frac{-1}{x} - x + C\right) .
\end{align*}
Now solve for the initial condition, $0 = \tan(-2+C)$ to get $C=2$ (or $2 +
\pi$, etc\ldots).  The solution we are seeking is, therefore,
\begin{equation*}
y = \tan \left(\frac{-1}{x} - x + 2 \right) .
\end{equation*}
}

\example{sep_coffeeexample}{Cooling a cup of coffee}{
Bob made a cup of coffee, and
Bob likes to drink coffee only once it will not burn him at 60
degrees.
Initially at time $t=0$ minutes,
Bob measured the temperature and the coffee was 89 degrees Celsius.
One minute later, Bob measured the coffee again and it had 85 degrees.
The temperature of the room (the ambient temperature) is 22 degrees.
When should Bob start drinking?}
{
Let $T$ be the temperature of the coffee, and let $A$ be the ambient (room) temperature.
\sword{Newton's law of cooling}\index{Newton's law of cooling} states that the rate at which the
temperature of the coffee is changing
is proportional to the difference between the
ambient temperature and the temperature of the coffee.  That is,
\begin{equation*}
\frac{dT}{dt} = k(A-T) ,
\end{equation*}
for some constant $k$.
For our setup $A=22$, $T(0) = 89$, $T(1) = 85$.
We separate variables and integrate (let $C$ and $D$ denote arbitrary
constants)
\begin{align*}
\frac{1}{T-A} \, \frac{dT}{dt} & = -k , \\
\ln (T-A) &= -kt + C , \qquad \text{(note that $T-A > 0$)} \\
T-A &= D\, e^{-kt} ,  \\
T &= A + D\, e^{-kt} .
\end{align*}
That is,
$T = 22 + D\, e^{-kt}$.  We plug in the first condition: $89 = T(0) = 22 +
D$,
and hence $D = 67$.  So
$T = 22 + 67\, e^{-kt}$.  The second condition says $85 = T(1) = 
22 + 67\, e^{-k}$.  Solving for $k$ we get
$k = - \ln \frac{85-22}{67} \approx 0.0616$.  Now we solve for the time $t$
that gives us a temperature of 60 degrees.  That is, we solve
$60 = 22 + 67 e^{-0.0616t}$ to get
$t = - \frac{\ln \frac{60-22}{67}}{0.0616} \approx 9.21$ minutes.  So Bob can
begin to drink the coffee at just over 9 minutes from the time Bob made
it.  That is probably about the amount of time it took us to calculate how long
it would take.
}\\

\example{egh4}{Finding all possible solutions}{
Find the general solution to $y' = \frac{-xy^2}{3}$ (including singular
solutions).}
{
First note that $y=0$ is a solution (a singular solution).
So assume that $y \not= 0$ and write
\begin{align*}
\frac{-3}{y^2} y' & = x , \\
\frac{3}{y} & = \frac{x^2}{2} + C , \\
y & = \frac{3}{\nicefrac{x^2}{2} + C}
= \frac{6}{x^2 + 2C}.
\end{align*}
}

\printexercises{exercises/14_04_exercises}


