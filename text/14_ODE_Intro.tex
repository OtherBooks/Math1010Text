\section{Introduction to differential equations}\label{sec:ODEintro}


\subsection*{Differential equations}

The laws of physics are generally written down as differential
equations.  Therefore, all of science and engineering use
differential equations to some degree.  Understanding
differential equations is essential to understanding almost anything you will
study in your science and engineering classes.
You can think of mathematics as the language of science, and
differential equations are one of the most important parts of this
language as far as science and engineering are concerned.  As an analogy,
suppose all your classes from now on were given in Swahili.  
It would be important to first learn Swahili, or you would have a very
tough time getting a good grade in your classes.

You have already seen many
differential equations without perhaps knowing about it.
And you have even solved simple
differential equations when you were taking calculus.
Let us see an example you may not have seen:
\begin{equation} \label{eq1}
\frac{dx}{dt} + x = 2 \cos t .
\end{equation}
Here $x$ is the \sword{dependent variable}\index{dependent variable} and $t$ is the
\sword{independent variable}\index{independent variable}.
Equation \eqref{eq1}
is a basic example of a \sword{differential equation}\index{differential equation}.  In fact it
is an example of a \sword{first order differential equation}\index{differential equation!first order}, since
it involves only the first derivative of the dependent variable.  This 
equation arises from Newton's law of cooling where the ambient
temperature oscillates with time.

\subsection*{Solutions of differential equations}

Solving the differential equation means finding $x$ in terms of $t$.  That
is, we want to find a function of $t$, which we will call $x$, such that when
we plug $x$, $t$, and $\frac{dx}{dt}$ into \eqref{eq1}, the equation holds.
It is
the same idea as it would be for a normal (algebraic) equation of just
$x$ and $t$.  We claim that
\begin{equation*}
x = x(t) = \cos t + \sin t
\end{equation*}
is a \sword{solution}\index{differential equation!solution}\index{solution}.
How do we check?  We simply plug $x$ into equation \eqref{eq1}!  First we
need to compute $\frac{dx}{dt}$.  We find that $\frac{dx}{dt} = 
-\sin t + \cos t$.  Now let us compute the left hand side
of \eqref{eq1}.
\begin{equation*}
\frac{dx}{dt} + x = 
(-\sin t + \cos t)
+
(\cos t + \sin t)
=
2\cos t .
\end{equation*}
Yay!  We got precisely the right hand side.
But there is more!
We claim
$x = \cos t + \sin t + e^{-t}$ is also
a solution.  Let us try,
\begin{equation*}
\frac{dx}{dt} = -\sin t + \cos t - e^{-t} .
\end{equation*}
Again plugging into the left hand side of \eqref{eq1}
\begin{equation*}
\frac{dx}{dt} + x = 
(-\sin t + \cos t - e^{-t}) +
(\cos t + \sin t + e^{-t})
= 2\cos t .
\end{equation*}
And it works yet again!

So there can be many different solutions.  In fact, for this equation all
solutions can be written in the form
\begin{equation*}
x = \cos t + \sin t + C e^{-t}
\end{equation*}
for some constant $C$.  See \ref{intro:plotsfig} for the graph of a
few of these solutions. 
We will see how we find these solutions
a few lectures from now.



\mtable{.5}{A few solutions of $\frac{dx}{dt} + x = 2 \cos t$.}{intro:plotsfig}{
\myincludegraphics[scale=.45]{figures/intro-plots-alt}
}

It turns out that solving differential equations can be quite hard.  
There is no general method that solves every differential equation.  We will
generally focus on how to get exact formulas for solutions of certain
differential
equations, but we will also spend a little bit of time
on getting approximate solutions.

For most of the course we will look at
\sword{ordinary differential equation}\index{ordinary differential equation}\index{differential equation!ordinary differential equation} (often abbreviated \sword{ODEs}, by which we mean that there
is only one independent variable and derivatives are only with respect to
this one variable.
If there are several independent variables, we will get
\sword{partial differential equations}\index{partial differential equation}\index{differential equation!partial differential equation} or \sword{PDEs}.

Even for ODEs, which are very well understood, it is not a simple question
of turning a crank to get answers.  
It is important to
know when it is easy to find solutions and how to do so.
Although in real applications you will
leave much of the actual calculations to computers, you
need to understand what they are doing.  It is often necessary
to simplify or transform your equations into something that a computer can
understand and solve.
You may need to make certain assumptions and changes in your
model to achieve this.

To be a successful engineer or scientist, you will be required to solve
problems in your job that you have never seen before.  It is important to
learn problem solving techniques, so that you may apply those techniques to
new problems.  A common mistake is to expect to learn some prescription for
solving all the problems you will encounter in your later career.  This
course is no exception.


\subsection*{Differential equations in practice}
\mtable{.7}{Mathematical modelling process}{}{
\myincludegraphics[scale=.5]{figures/1-1-fig}
}

So how do we use differential equations in science and engineering?  
First, we have some \sword{real world problem} we wish
to understand.
We make some simplifying assumptions and create a \sword{mathematical
model}\index{mathematical model}
That is, we translate the real world situation into a
set of differential equations.
Then we apply mathematics to get some sort of a \sword{mathematical
solution} to the model.
There is still something left to do.  We have to interpret the results.
We have to figure out what the mathematical solution says about the real world
problem we started with.

Learning how to formulate the mathematical model and how to interpret the
results is what your physics and engineering classes do.  In this
course we will focus mostly on the mathematical analysis.  Sometimes we will
work with simple real world examples, so that we have some intuition and
motivation about what we are doing.

Let us look at 
an example of this process.
One of the most basic differential equations
is the standard \sword{exponential growth model}\index{exponential growth model}
Let $P$ denote the population 
of some bacteria on a Petri dish.  We assume that there is enough food
and enough space.  Then the rate of growth of bacteria is proportional
to the population---a large population grows quicker.  Let $t$ denote
time (say in seconds) and $P$ the population.  Our model
is
\begin{equation*}
\frac{dP}{dt} = kP ,
\end{equation*}
for some positive constant $k > 0$.\\

\example{ex_bacgrowth}{Model for bacterial growth}{
Suppose there are 100 bacteria at time 0 and 200 bacteria 10 seconds later.
How many bacteria will there be 1 minute from time 0 (in 60 seconds)?}
{First we have to solve the equation.  We claim that a solution is given by
\begin{equation*}
P(t) = C e^{kt} ,
\end{equation*}
where $C$ is a constant.  Let us try:
\begin{equation*}
\frac{dP}{dt} = C k e^{kt} = k P .
\end{equation*}
And it really is a solution.

OK, so what now?  We do not know $C$ and we do not know $k$.  But we know
something.  We know $P(0) = 100$, and we also know 
$P(10) = 200$.  Let us plug these conditions in and see what happens.
\begin{align*}
& 100 = P(0) = C e^{k0} = C ,\\
& 200 = P(10) = 100 \, e^{k10} .
\end{align*}
Therefore, $2 = e^{10k}$ or $\frac{\ln 2}{10} = k \approx 0.069$.
So we know that
\begin{equation*}
P(t) = 100 \, e^{(\ln 2) t / 10} \approx 100 \, e^{0.069 t} .
\end{equation*}
At one minute, $t=60$, the population is $P(60) = 6400$.  See Figure
\ref{intro:plotbactfig}.

\mtable{0.65}{Bacteria growth in the first 60 seconds.}{intro:plotbactfig}{%
\myincludegraphics[scale=.4]{figures/intro-plotbact}
}

Let us talk about the interpretation of the results.  Does our solution
mean that
there must be exactly 6400 bacteria on the plate at 60s?  No!  We made
assumptions that might not be true exactly, just approximately.
If our assumptions are reasonable,
then there will be approximately 6400 bacteria.
Also, in real life $P$ is a
discrete quantity, not a real number.  However, our model has no problem saying
that for example at 61 seconds, $P(61) \approx 6859.35$.
%Obviously there 
%are either 6859 bacteria or 6860 bacteria.
}\\

Normally, the $k$ in $P' = kP$ is known,
and we want to solve
the equation for different \sword{initial conditions}\index{initial condition}\index{differential equation!initial condition}.
What does that mean?
Take $k=1$ for simplicity.  Now suppose we want to solve the equation
$\frac{dP}{dt} = P$ 
subject to $P(0) = 1000$ (the initial condition).
Then the solution turns out to be (exercise)
\begin{equation*}
P(t) = 1000 \, e^t .
\end{equation*}

We call $P(t) = C e^t$ \sword{general solution}\index{general solution}\index{solution!general},
as every solution
of the equation can be written in this form for some constant $C$.  You
will need an initial condition to find out what $C$ is, in order to find the
\sword{particular solution}\index{particular solution} we are looking for.  Generally, when we say
``particular solution,'' we just mean some solution.

\medskip

Let us get to what we will call the four fundamental equations.
These equations appear very often and it is useful to just memorize what
their solutions are.
These solutions
are reasonably easy
to guess by recalling properties of exponentials, sines, and cosines.
They are also simple to check, which is something that you should always do.
There is no need to wonder if you have remembered the solution correctly.
%You will be
%required to know these and be able to solve these.
%I promise at least one or
%two of these will appear on EVERY test.  You should just memorize their
%solutions.  You can and should always check that you have the
%right solution, so that makes the memorization easier.  It is in fact not
%hard to guess these solutions.

\medskip

First such equation is,
\begin{equation*}
\frac{dy}{dx} = k y ,
\end{equation*}
for some constant $k > 0$.
Here $y$ is the dependent and $x$ the independent variable.
The general solution for this equation is
\begin{equation*}
y(x) = C e^{kx} .
\end{equation*}
We have already seen that this function is a solution above with different
variable names.

\medskip

Next,
\begin{equation*}
\frac{dy}{dx} = -k y ,
\end{equation*}
for some constant $k > 0$.
The general solution for this equation is
\begin{equation*}
y(x) = C e^{-kx} .
\end{equation*}

\medskip

\noindent{\bf Exercise:} Check that the $y$ given is really a solution to the equation.

\medskip

Next, take the
\sword{second order differential equation}\index{second order differential equation}
\begin{equation*}
\frac{d^2y}{{dx}^2} = -k^2 y ,
\end{equation*}
for some constant $k > 0$.
The general solution for this equation is
\begin{equation*}
y(x) = C_1 \cos(kx) + C_2 \sin(kx) .
\end{equation*}
Note that
because
we have a second order differential equation,
we have two constants in our general solution.

\medskip

\noindent{\bf Exercise:} Check that the $y$ given is really a solution to the equation.

\medskip

And finally, take the second order differential equation
\begin{equation*}
\frac{d^2y}{{dx}^2} = k^2 y ,
\end{equation*}
for some constant $k > 0$.
The general solution for this equation is
\begin{equation*}
y(x) = C_1 e^{kx} + C_2 e^{-kx} ,
\end{equation*}
or
\begin{equation*}
y(x) = D_1 \cosh(kx) + D_2 \sinh(kx) .
\end{equation*}

For those that do not know, $\cosh$ and $\sinh$ are defined by
\begin{align*}
\cosh x &= \frac{e^{x} + e^{-x}}{2} , \\
\sinh x &= \frac{e^{x} - e^{-x}}{2} .
\end{align*}
These functions are sometimes easier to
work with than exponentials.  They have some nice familiar
properties such as
$\cosh 0 = 1$, $\sinh 0 = 0$, and $\frac{d}{dx} \cosh x = \sinh x$ (no that is
not a typo)
and $\frac{d}{dx} \sinh x = \cosh x$.

\medskip

\noindent{\bf Exercise:} Check that both forms of the $y$ given are
really solutions to the equation.

\medskip

An interesting note about $\cosh$:  The graph of $\cosh$ is the exact shape
a hanging chain will make.  This shape is called
a \sword{catenary}\index{catenary}
Contrary to popular belief this is not a
parabola.  If you invert the graph of $\cosh$ it is also the ideal arch for
supporting its own weight.
For example, the gateway arch in Saint Louis is an inverted graph of
$\cosh$---if it were just a parabola it might fall down.  The formula
used in the design is
inscribed inside the arch:
\begin{equation*}
y = -127.7 \; \textrm{ft} \cdot \cosh({x / 127.7  \; \textrm{ft}}) + 757.7 \;
\textrm{ft} .
\end{equation*}

\printexercises{exercises/14_01_exercises}
